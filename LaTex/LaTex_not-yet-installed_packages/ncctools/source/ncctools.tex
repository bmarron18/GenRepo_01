\documentclass[11pt]{article}
\usepackage[colorlinks=true,filecolor=blue]{hyperref}
\usepackage{amsmath}
\usepackage{desclist}
\usepackage{tocenter}
\reversemarginpar
\settowidth\marginparwidth{\tt afterpackage}
\ToCenter[fm]{130mm}{220mm}

\title{NCCTOOLS}
\author{by Alexander I. Rozhenko}
\date{Release 3.5}

\def\|{\verb|}
\newcommand*\Package[1]{%
  \par\bigskip\noindent\leavevmode
  \marginpar{\hfill\href{#1.pdf}{\underbar{\tt#1}}}\ignorespaces
}
\newcommand*\Pack[1]{{\tt#1}}
\begin{document}
\maketitle

The `ncctools' collection consists of a number of packages
extracted from NCC style (developed by Alexander I.~Rozhenko in
1992--1996 under \LaTeX-2.09) while re-implementation it for
\LaTeXe. Many new packages were also added later.

The collection now contains 25 packages providing the following:

\Package{afterpackage} Implements \|\AfterPackage| command specifying
additional commands which should be applied to a package after
its loading. The customization commands are specified without
pre-loading packages they concerned to and are applied right
after loading the respective packages.

\Package{dcounter} Dynamic counters. The counter declared as
dynamic is really created at the first use and receives at that
moment the count style established by the \|\countstyle| command.
The special use of \|\counstyle| command with optional parameter
allows modify the subordination of existing counter. For example,
while using the book class you can reject the subordination of the
section counter to the chapter counter and re-subordinate figures,
tables and equations to sections. The package is used in the
\Pack{nccthm} package.

\Package{desclist} Implements the \|desclist| environment. It is
considered as an improvement of the \|description| environment.
The appearance of item markers is easy customizable on the fly. An
optional parameter allows set a marker prototype for calculation
of hang indentation skip. The \|description| environment is
redefined to use an optional parameter also.

\Package{extdash} The package implements the commands,
\|\Hyphdash|, \|\Endash|, \|\Emdash|, and their star-forms, to
control hyphenation of compound words and ordinary words dashed by
em-dash. You can also use the shortcuts
\begin{center}
\|\-/|\quad \|\=/|\quad \|\--|\quad \|\==|\quad\|\---|\quad\|\===|
\end{center}
instead. You can also decrease the length of em-dash by the
\|cyremdash| option to satisfy the Russian typesetting rules.

\Package{manyfoot} The package implements a command,
\|\newfootnote|, that adds footnote levels to the standard LaTeX's
footnote mechanism. Footnotes of every additional level are
automatically grouped together on a \LaTeXe\ output page and are
separated from another levels by the special vertical space and
(maybe) rule. You can customize the typesetting style of
additional footnotes choosing between ordinary footnotes and
run-in paragraph footnotes (useful for critical editions). Service
command \|\DeclareNewFootnote| simplifies creation of new footnote
levels with automatic footnote numbering. The possibility of
customization inter-level footnote rules is allowed.

\Package{mboxfill} The package introduces \|\mboxfill| command
filling a free space with a pattern. All leader types are supported.
Width of pattern can be specified by the same manner as in
the \|\makebox| command.

\Package{nccbbb} Implementation of poor Black Board Bold symbols.
Ported from old NCC-\LaTeX. It is useless in modern \LaTeX\ but
kept just in case.

\Package{nccboxes} Additional boxes from NCC-\LaTeX. The \|\jhbox|
and \|\jvbox| horizontally and vertically align a body with
respect to a prototype. The \|\jparbox| vertically aligns
paragraph box with respect to a prototype. The \|\addbox| adjusts
height and depth of box. The \|\pbox| is a simple version of
one-column table. It is independent on \|\arraystretch| value. The
\|\cbox| is intended for design of fancy headers in tables.

\Package{ncccomma} Implements the smart comma in math mode working
as an ordinary character if a decimal character goes after it.
Otherwise, the math comma works as a punctuation mark.

\Package{ncccropbox} Implements the \|\cropbox| command preparing
a box with crop marks at its corners looking like angles. Angle
parameters are customizable.

\Package{ncccropmark} Implements the \|\cropmark| command
producing crop box around page text area (header area, footer area
and marginal notes are optionally taken into consideration). The
\|\cropmark| command is useful as a parameter of the \|\watermark|
commands form the \Pack{watermark} package. It accurately
interprets current state of two-column, two-side, and
reverse-margin modes.

\Package{nccfancyhdr} Absolutely new implementation of
functionality of the \Pack{fancyhdr} package. It is more
transparent, simple, and non-aggressive (redefining of standard
page styles is optional). Using the package with names of standard
page styles as options, you can easy decorate your document with
header/footer rules. For example, the command
\begin{center}
\|\usepackage[headings]{nccfancyhdr}|
\end{center}
sets the \|headings| page style and provides it with the
decorative rule at the header. Header width control is improved
with two commands, namely \|\extendedheaders| (extended upon
marginal notes) and \|\normalheaders|. The \|\thispagestyle|
command correctly works with the fancy page style (in fancyhdr, it
didn't work because of use of global definitions). A new page
style can be easy created with the help of the \|\newpagestyle|
command and fancy mark commands.

\Package{nccfloats} Wraps \LaTeX\ floats with service commands
\|\fig|, \|\tabl|, \|\figs|, \|\tabls|, introduces the \|\minifig|
and \|\minitabl| commands preparing figure and table in a minipage
with possible use of \|\caption| within, and \|\sidefig| and
\|\sidetabl| used for placement of minifloats next to surrounding
text on the outer side of page.

\Package{nccfoots} The package implements commands for generating
footnotes with manual marks. For example, to mark footnote by star
you can write
\begin{center}
 \|\Footnote{$*$}{Footnote text}|.
\end{center}

\Package{nccmath} Extension of the \Pack{amsmath} package.  Its
main aim is to combine \AmS's typesetting of display equations and
NCC-\LaTeX's one. In \Pack{amsmath}, the \|eqnarray| environment
leaves unchanged. This package redefines \|eqnarray| to allow
using of \Pack{amsmath} tag control features and display breaks.
Inter-column distance in \|eqnarray| is reduced to the distance
typical for relation operations.  All columns are prepared in the
\|\displaystyle|.  A new \|darray| environment is a mix of the
\AmS's \|aligned| environment and \LaTeX's \|array| environment.
It is typed out in the same way as the \|aligned| environment but
has columns definition parameter as in \|array| environment. The
use of column specifications is restricted to the necessary
commands only: \|l|, \|c|, \|r|, \|@|, and \|*| are allowed. The
implementation has no conflicts with the packages redefining
arrays. The \|fleqn| and \|ceqn| environments allow dynamically
change the alignment of display formulas to flushed left or to
centered alignment. Some additional commands are introduced also.

\Package{nccparskip} Useful for documents with non-zero skips
between paragraphs. In this case, the additional vertical space
inserted by lists is unlikely. The package provides identical
distance between all paragraphs except sectioning markup commands. It
redefines control list commands and suppress \|\topskip|,
\|\partopskip|, and \|\itemsep| in lists. As a result, the
distance between ordinary paragraphs and paragraphs prepared by
lists is the same. The  \|\SetParskip{distance}| command controls
this distance.

\Package{nccpic} Envelop for the \Pack{graphicx} package. It
customizes graphics extensions list for dvips driver. You need not
specify a graphics file extension when use the \|\includegraphics|
command. Depending on a dvi-driver specified, a graphics file with
an appropriate extension is searched.  So, you only need to create
a number of versions of a graphics file in different formats (for
example, `.bmp' for dvips or Yap and `.png' for pdftex).  After
that you can produce resulting `.ps' and `.pdf' file without any
changes in the source file. The recommended storage for graphics
files is the `graphics/' subdirectory of the directory the `.tex'
file is translated. Some additional commands are introduced also.

\Package{nccrules} Implements two commands, \|\dashrule| and
\|\dashrulefill|, which compose dashed (multi)lines. Two footnote
rule generation commands, \|\newfootnoterule| and
\|\newfootnotedashrule|, are useful in conjunction with the
\|manyfoot| package.

\Package{nccsect} Extension of \LaTeX's section, caption, and
toc-entries generation technique.  The package contains many
improvements in comparison with the base \LaTeX's implementation.
The most interesting of them are:
\begin{itemize}
\item simple declaring of sections of any level (including
sections of 0th level and captions for floats);

\item user-controlled typeout for display sections (user can
select one of the following typeout styles: \|hangindent|,
\|hangindent*|, \|parindent|, \|parindent*|,
\|hangparindent|, \|hangparindent*|, \|center|, and 
\|centerlast|);{\sloppy\par}

\item new section styles can be easy constructed with the help of
two style definition commands;

\item customizing of section or caption tag by the manner similar
to \AmS\ equation tag;

\item simple declaring of toc-entries using prototypes for
calculation of hang indentations;

\item \|\numberline| command newer overlaps the text going after;

\item \|\PnumPrototype| command is used for calculation of right
margin in table of contents;

\item different captions for different float types;

\item simple handling of new types of floats (after registration of
a new float in the package, you can declare a caption and toc-entry
for it; be sure that the \|\chapter| command will automatically
produce a vertical skip in a toc for the new float also).
\end{itemize}

\Package{nccstretch} Implements the \|\stretchwith| command that
stretches a text inserting something between every pair of
neighbour tokens.

\Package{nccthm} Yet another extension to the \|\newtheorem|
command. The following orthogonal properties of theorems are used:
\begin{desclist}{\it}{\/}[numbering mode]
\item[numbering mode] is standard or \textit{apar\/} (a number
before header);

\item[theorem type] defines an appearance of a theorem (what fonts
are used for title, comment, and body). The `theorem' and
`remark' types a predefined;

\item[indent style] margin, nomargin, indent, noindent (selected
in package options);

\item[break style] do break after header or not? Hardcoded when a
new theorem is declared. Can be overridden on the fly for concrete
math statement.
\end{desclist}

Easy customization of spacing and commands inserted at the end of
theorem headers.
Two types of Q.E.D. symbol (white and black).  The \|\proof|
command is introduced.  The \|\newtheoremtype| command allows
create new theorem types.  The \|\like|\textit{type\/} command is
automatically created for every new type.  It simplifies typeout
of rare math statements.  Instead of creation of a new theorem
environment you can use
\begin{center}
\|\liketheorem{Title}{Number}[Comment]|
\end{center}
for it, or \|\likeremark|, etc.  Every theorem environment and
theorem type can be redefined with \|\renewtheorem| and
\|\renewtheoremtype|.  Counters of all theorem-like environments
are dynamic (the \Pack{dcounter} package is used). So, they are
created at the first use. This is very helpful for package writer.
A number of theorem-like environments can be created in a package
or class, and a user only selects the count style for them (in
simple case this can be done with only one \|\countstyle| command
in the document preamble).


\Package{textarea} Allows expand the text area of a document
on other areas: footer, header, and margin. There are two
ways for such expansion~--- temporary expansion of a current
page on the header using a special negative vertical skip,
or global change of page layout parameters with possibility
of later restoration them to the original values.
All these actions should be applied at the beginning of page.

\Package{tocenter} Provides two commands, namely \|\ToCenter| and
\|\FromMargins|, simplifying the customization of page layout. It
is now easy to change text width and height and center the text
area on the page (the header, footer, and marginal note fields are
optionally taken into consideration).  Other way is like-Word
declaring of paper layout: you set margins from the left, right,
top, and bottom edges of page.

\Package{topsection} Implements an unnumbered top-level section
(\|\chapter| in books and reports or \|\section| in articles) in
class-independent way. Such a section is used as a title for
standard document sections such as table of contents, index,
bibliography, etc. Using this command allows simplify the
definition or redefinition of standard document environments.

\Package{watermark} Provides watermarks on output pages. A
watermark is some text or picture printed at the background of
paper. A watermark is usually stored in the page header. This way
is inconvenient: it needs redefinition of page style commands. We
use another method that is orthogonal to page mark commands and
needs no redefinition of page marks. The left and right watermarks
are allowed. Temporary \|\thiswatermark| acts on the current page
only.  Using this way it is easy to replace a page header on a
page by your own page header with the \|\thispageheading| command.

\end{document}
