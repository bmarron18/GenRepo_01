%%%%%%%%%%%%%%%%%%%%%%%%%%%%%%%%%%%%%%%%%
% Hwk1
% 2016SoE017
% STAT_564_Linear-Regression
% 
% Date: 04 Oct 2016
% Author: bmarron
%
%%%%%%%%%%%%%%%%%%%%%%%%%%%%%%%%%%%%%%%%%

%----------------------------------------------------------------------------------------
%	PACKAGES AND OTHER DOCUMENT CONFIGURATIONS
%----------------------------------------------------------------------------------------

\documentclass[twoside]{article}	%use

\usepackage{lipsum} % Package to generate dummy text throughout this template
\usepackage{csquotes}
\usepackage{apacite}
\usepackage[english]{babel}
\usepackage{amsmath}
\usepackage{amsthm}
\usepackage{amssymb}
\usepackage{pdfpages}
\usepackage{verbatim}
\usepackage{bigfoot}
\usepackage{multirow}


\usepackage[sc]{mathpazo} % Use the Palatino font and the Pazo fonts for math
\usepackage[T1]{fontenc} % Use 8-bit encoding that has 256 glyphs
\linespread{1.05} % Line spacing - Palatino needs more space between lines
\usepackage{microtype} % Slightly tweak font spacing for aesthetics

\usepackage[hmarginratio=1:1,top=32mm,columnsep=20pt]{geometry} % Document margins
\usepackage{multicol} % Used for the two-column layout of the document
\usepackage[hang, small,labelfont=bf,up,textfont=it,up]{caption} % Custom captions under/above floats in tables or figures
\usepackage{booktabs} % Horizontal rules in tables
\usepackage{float} % Required for tables and figures in the multi-column environment - they need to be placed in specific locations with the [H] (e.g. \begin{table}[H])
%\usepackage{hyperref} % Interferes with cite.sty 

%\usepackage{lettrine} % The lettrine is the first enlarged letter at the beginning of the text
\usepackage{paralist} % Used for the compactitem environment which makes bullet points with less space between them

\usepackage{abstract} % Allows abstract customization
\renewcommand{\abstractnamefont}{\normalfont\bfseries} % Set the "Abstract" text to bold
\renewcommand{\abstracttextfont}{\normalfont\small\itshape} % Set the abstract itself to small italic text

\usepackage{titlesec} % Allows customization of titles
\renewcommand\thesection{\Roman{section}} % Roman numerals for the sections
\renewcommand\thesubsection{\arabic{subsection}} % Arabic numerals for subsections
\titleformat{\section}[block]{\large\scshape\centering}{\thesection.}{1em}{} % Change the look of the section titles
\titleformat{\subsection}[block]{\large}{\thesubsection.}{1em}{} % Change the look of the section titles

\usepackage{fancyhdr} % Headers and footers
\pagestyle{fancy} % All pages have headers and footers
\fancyhead{} % Blank out the default header
\fancyfoot{} % Blank out the default footer
\fancyhead[R]{\date{\today}}
\fancyfoot[R]{\thepage} % Custom footer text
%\fancyhead[C]{Running title $\bullet$ November 2012 $\bullet$ Vol. XXI, No. 1} % Custom header text
%\fancyfoot[RO,LE]{\thepage} % Custom footer text

%----------------------------------------------------------------------------------------
%	TITLE SECTION
%----------------------------------------------------------------------------------------

\title{\vspace{-15mm}\fontsize{14pt}{10pt}\selectfont\textbf{Homework 1 (STAT 564)}} % Article title

\author{
\large
\textsc{Bruce Marron} \\ %\thanks{A thank you or further information}\\[2mm] % Your name
\normalsize Portland State University \\ % Your institution
\vspace{-5mm}
}
\date{}

%----------------------------------------------------------------------------------------
\begin{document}
\maketitle % Insert title
\thispagestyle{fancy} % All pages have headers and footers

%----------------------------------------------------------------------------------------
%	ABSTRACT
%----------------------------------------------------------------------------------------

%\begin{abstract}
%\end{abstract}

%----------------------------------------------------------------------------------------
%	ARTICLE CONTENTS
%----------------------------------------------------------------------------------------
% \begin{multicols}{2} (Two-column layout throughout the entire main article text, if desired)
% \end{multicols} (place this just before REFERENCE LIST)

% \noindent  (as needed, place in front of new paragraph)
% Table ~\ref{tab:bm1}  (as needed to reference Tables in text)
% Figure ~\ref{fig:bm2} (as needed to reference Figures in text)
% \verb!FormatProcessedLevel3_MFS_dataset1.csv! (as needed for small verbatim)

% \begin{verbatim}   (for large verbatim)
% \end{verbatim}

\subsection{Introduction}

\subsection{Methods and Materials}


 



%----------------------------------------------------------------------------------------
%	REFERENCE LIST
%----------------------------------------------------------------------------------------
\newpage
\bibliographystyle{/usr/local/share/texmf/tex/latex/apacite/apacite}
\bibliography{/home/bmarron//Desktop/BibTex/My_Library_20160725}


\newpage
\vspace*{5cm}
\begin{center}
\LARGE \textbf{Appendix}
\end{center}
%\includepdf[pages=1-21]{figures/LOG1_FormatProcessing_MFS_dataset1.pdf}
%\includepdf[pages=1-5]{figures/LOG2_FormatProcessing_MFS_dataset1.pdf}


\end{document}
