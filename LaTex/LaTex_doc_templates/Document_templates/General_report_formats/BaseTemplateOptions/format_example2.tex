% This is a simple LaTex sample document that gives a submission format
%   for IEEE PAMI-TC conference submissions.  Use at your own risk.

% Make two column format for LaTex 2e.\
\documentclass[12pt,twocolumn]{article} %,twocolumn
\usepackage[dvips]{graphicx,graphics}
\usepackage{times,amsmath,amsfonts}

% Use following instead for LaTex 2.09 (may need some other mods as well).
% \documentstyle[times,twocolumn]{article}

% Set dimensions of columns, gap between columns, and paragraph indent
\setlength{\textheight}{9.2in} \setlength{\textwidth}{6.875in}
%\setlength{\columnsep}{0.3125in} \setlength{\topmargin}{0in}
\setlength{\headheight}{0in} \setlength{\headsep}{-1in}
\setlength{\parindent}{1pc}
\setlength{\oddsidemargin}{-.1875in}  % Centers text.
\setlength{\evensidemargin}{-.1875in}

% Add the period after section numbers.  Adjust spacing.
\newcommand{\Section}[1]{\vspace{-8pt}\section{\hskip -1em.~~#1}\vspace{-3pt}}
\newcommand{\SubSection}[1]{\vspace{-3pt}\subsection{\hskip -1em.~~#1}
        \vspace{-3pt}}
\newcommand{\bqn}{\begin{eqnarray}}
\newcommand{\eqn}{\end{eqnarray}}
\newcommand {\diff}[1] {\frac{\partial}{\partial #1}}
\newcommand{\jacob}[3]{\frac{\partial^2 #3}{\partial #1 \partial #2}}
\newcommand{\der}[2]{\frac{\partial #2}{\partial #1}}
\begin{document}

% Make title bold and 14 pt font (Latex default is non-bold, 16pt)
\title{Stat 312: Lecture 26\\
Contingency Tables}
% For single author (just remove % characters)
\author{Moo K. Chung\\
mchung@stat.wisc.edu}
% For two authors (default example)
\maketitle \thispagestyle{empty}
\subsection*{Concepts}
\begin{enumerate}
\item Two-way contingency table: suppose there are $I$ populations
and each population is classified into $J$ categories. Let
$n_{ij}$ be the number of {\em observed} elements in population
$i$ which fall into category $j$. We denote $n_{\bullet j} =
\sum_{i=1}^I n_{ij}$ and $n_{i \bullet} = \sum_{j=1}^J n_{ij}$.


\item Testing for {\em homogeneity}. Let $p_{ij}$ be the
proportion of the elements in population $i$ which fall into
category $j$. Note that $\sum_{j=1}^J p_{ij}=1$. We want to test
if the proportions in the different categories are the same for
all populations, i.e.
$$H_0: p_{1j} = p_{2j} = \cdots = p_{Ij} \mbox{ for all } j.$$

\item The {\em expected} number of element $\mathbb{E} N_{ij}$ in
population $i$ which falls into category $j$. $\mathbb{E}
N_{ij}=n_{i\bullet}p_{ij}$. Under the null hypothesis,
$p_{ij}=\cdots=p_{Ij}=p_j$. So $\mathbb{E}N_{ij}=n_{i
\bullet}p_j$. We estimate $p_j$ by pooling $I$ samples together.
$\hat p_j = n_{\bullet j}/n$.

\item Test statistic:
$$\chi^2 = \sum_{i,j}
\frac{(n_{ij}-\mathbb{E}N_{ij})^2}{\mathbb{E}N_{ij}} \sim
\chi^2_{(I-1)(J-1)}.$$
\end{enumerate}

\subsection*{In-class problems}
{\em Example.} Suppose that 20 out of 50 females and 10 out of 40
males are depressed in a sample. Determine if the frequency of
depression is related to sex.\\
 {\em Solution.} Following the notations above, we are interested
 in testing $H_0: p_{11}=p_{21}, p_{12}=p_{22}$. $n=90$. Under $H_0$, we
 estimate $\hat p_1=(20+10)/(20+10+30+30) = 1/3$ and $\hat p_2=1-p_1=2/3$.
The test statistic
 value is $\chi^2 = $\\
The cutoff value $\chi^2_{\alpha,1}$ for $\alpha$-level test can
be computed from $N(0,1)$.
$$\alpha=P(\chi^2_1 > \chi^2_{\alpha,1}) = P(Z^2>
\chi^2_{\alpha,1})=2P(Z > \chi_{\alpha,1}).$$ So
$\chi^2_{\alpha,1}=Z^2_{\alpha/2}$. For $\alpha=0.05$,
$\chi^2_{\alpha,1}=1.96^2$.
\subsection*{Self-study problems}
Example 14.13.
\end{document}
