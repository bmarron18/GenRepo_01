%The following LaTeX file is an example of an acceptable paper for the 
%Communications in Statistics.

%cisform.tex
%
\documentclass[12pt]{article}
\setlength{\oddsidemargin}{0in}
\setlength{\evensidemargin}{0in}
\setlength{\topmargin}{-.5in}
\setlength{\headsep}{0in}
\setlength{\textwidth}{6.5in}
\setlength{\textheight}{8.5in}
\def\refhg{\hangindent=20pt\hangafter=1}
\def\refmark{\par\vskip 2mm\noindent\refhg}
\def\refhg{\hangindent=20pt\hangafter=1}
\def\refmark{\par\vskip 2mm\noindent\refhg}
\def\refhg{\hangindent=20pt\hangafter=1}    %20pt
\def\refhgb{\hangindent=10pt\hangafter=1}
\def\refmark{\par\vskip 2mm\noindent\refhg}
\renewcommand{\baselinestretch}{1.5}
\begin{document}
%
%
\centerline{EXAMPLE}
\vskip 3mm


\noindent OPTIMUM SCORES UNDER ORDER CONSTRAINTS
IN CONTINGENCY TABLES
\vskip 3mm


\vskip 5mm
\noindent A. Rahulji Parsa and William B. Smith

\noindent Department of Statistics

\noindent Texas A\&M University

\noindent College Station, Texas \ 77843-3143

\noindent smith@stat.tamu.edu

\vskip 3mm
\noindent Key Words: contingency tables; structural categories; optimal scoring.
\vskip 3mm


\noindent ABSTRACT


%
Methods are developed for analyzing contingency tables which have
ordered categories, {\it a priori}.  An exact representation is obtained
when differences in the scores of the categories are known  (called, strong
structure).  When   only category ordering is known (weak structure) several techniques are reduced to solutions constrained optimization problems.  Numerical comparisons with other techniques are given in each case.
%
\vskip 4mm


\noindent 1.   INTRODUCTION AND NOTATION



The problem of assigning scores to the rows and columns of a
contingency table is an old one.  Using canonical analysis is one
method of solution, while correspondence analysis provides another
method.  Interest in correspondence analysis has been increasing in
recent years, mainly due to the works of Ben\'ezecri and Lebart and the
books by Greenacre (1984), Bradley et al. (1962), Goodman (1979) and Haberman (1974) 
have demonstrated the close relationship between canonical analysis, 
correspondence analysis and principal 

In this paper we propose a method based on correspondence analysis to
find optimum scores for rows and columns of an $I\times J$
contingency table when there is either a ``strong'' or a ``weak''
structure on the categories.  In a ``strong'' structure, the differences in the scores of the categories are specified, while in a ``weak'' structure only a relative order of the categories is specified.

Let $N_{ij}$ be the frequencies ($i=1,\ldots, I; j=1,\ldots, J$) in
an $I\times J$ table.
Following Greenacre (1) we define the following quantities.
%
$$
P=\left[ p_{ij}\right] \hbox{ of order } I\times J, \hbox{ where }
p_{ij} = n_{ij}/n \eqno(1.1)
$$
%
and
$$
n=\sum_i\sum_jn_{ij}; \quad n_{i.} = \sum_jn_{ij}; \quad n_{.j} =
\sum_i n_{ij}. \eqno(1.2)
$$
%
Let 
$$
r=Pe, \quad c = P'e\eqno(1.3)
$$
%
where $e$ denotes a column vector of unit elements of an appropriate
order.  Let $D_r$, $D_c$ denote diagonal matrix with 
elements given by those of $r$ and $c$, respectively.  Then
%
$$
E=rc'=\left[\frac{n_{i.}n_{.j}}{n}\right]. \eqno(1.4)
$$
%
%
where $A$ is of order $I\times k$, $\Lambda$ is a diagonal matrix of
order $k\times k$ with diagonal elements $\lambda_1\geq \lambda_2\geq
\ldots\geq \lambda_k$, $B$ is of order $J\times k$, $k$ is the smaller
of $(I-1)$ and $(J-1)$, and $\lambda^2_{\alpha}(\alpha =
1,2,\ldots,k)$ are the eigenvalues of $(P-E)(P-E)'$.  It is a
``generalized'' SVD because   $A$ and $B$ are normalized  with
respect to metrics $D^{-1}_r$ and $D_c^{-1}$, that is, 
\vskip 3mm


\noindent 2. STRONG ORDER CONSTRAINTS


Order constraints are of at least two types; one where the
differences in scores of the categories is exactly specified and the
other where the  categories is exactly specified and the other where the  categories are ordered. Since order constraints are in terms of differences
of scores, they can be expressed in terms of contrast matrices of the
type (illustrated for $I=5$, say)
%
$$
\begin{array}{l}
\hbox{(a)} \left[\matrix {1 & -1 & 0 & 0 & 0 \cr
                          1 & 0 & -1 & 0 & 0 \cr 
                          1 & 0 & 0 & -1 & 0 \cr
                          1 & 0 & 0 & 0 & -1\cr}\right]\cr \\
%
\hbox{(b)}\left[\matrix  {1 & -1 & 0 & 0 & 0 \cr
                          0 & 1 & -1 & 0 & 0 \cr
                          0 & 1 & 1 & -1 & 0 \cr
                          0 & 1 & 0 & 0 & -1 \cr}\right]\cr \\
%
\hbox{(c)}\left[\matrix  {1 & -1 & 0 & 0 & 0 \cr
                          0 & 1 & -1 & 0 & 0 \cr
                          0 & 0 & 1 & -1 & 0 \cr 
                          0 & 0 & 1 & 0 & -1 \cr}\right]\cr \\
%
\hbox{(d)}\left[\matrix  {1 & -1 & 0 & 0 & 0 \cr 
                          0 & 1 & -1 & 0 & 0 \cr
                          0 & 0 & 1 & -1 & 0 \cr
                          0 & 0 & 0 & 1 & -1 \cr}\right]
\cr \\
\end{array}
$$
%

We will assume that the complete order of all categories is known and
without loss of generality use the matrix $0_1$ of the form (d) above
%

%eject
\vskip 3mm


\noindent 3.  ILLUSTRATION


As an illustration, consider the data from Goodman (3).

%
\begin{center}
Table I: Cross-classification of 135 Women According to their
Periodontal Condition and Calcium Intake Level.
\end{center}

\begin{center} 
\begin{tabular}{ccccccc}
   & & \multicolumn{5}{c}{Level} \\ \hline
          &   & 1 & 2 & 3 & 4 & Total \\ \hline
          & A & \ 5 & \ 3 & 10 & 11 & 29 \\
Condition & B & \ 4 & \ 5 & \ 8 & \ 6 & 23 \\
          & C & 26 & 11 & \ 3 & \ 6 & 46 \\
          & D & 23 & 11 & \ 1 & \ 2 & 37 \\ \hline
   & Total & 58 & 30 & 22 & 25 & 135
\end{tabular}
\end{center}

The order constraint is $0_1F_1=\Delta_1$ where 
%
$$
0_1=\left[\matrix {1 & -1 & 0 & 0 \cr
                  0 & 1 & -1 & 0 \cr 
                  0 & 0 & 1 & -1 \cr}\right]; 
\Delta_1=\left[\matrix {.92\cr
                       .92\cr
                       .92\cr
                       .92\cr}\right]\eqno(3.1)
$$
%
From the SVD of $R-e\, c'$, without considering the constraints
(3.1), the optimal $F_1$ comes out as
%
\begin{center}
Table II: Treatment by Categories Matrix of Response Frequencies.
\end{center}

\begin{center} 
\begin{tabular}{ccccccc}
%
          & A & B & C & D & E & Total \\ \hline
A   & \ 9 & \ 5 & \ 9 & 13 & \ 4 & 40 \\
B   & \ 7 & \ 3 & 10 & 20 & \ 4 & 44 \\
C   & 14 & 13 & 6 & \ 7 & \ 0 & 40 \\
D   & 11 & 15 & 3 & \ 5 & \ 8 & 42 \\
E   & \ 0 & \ 2 & 10 & 30 & \ 2 & 44 \\ \hline
Total & 41 & 38 & 38 & 75 & 18 & 210 \\
\end{tabular}

\vspace{.3in}
A = excellent, B = good, C = Fair, D = Poor, E = Terrible.
\end{center}


\noindent 4. SUMMARY


Procedures were presented for developing scores which incorporate the
natural order in the objects (or categories) that characterize the
contingency table.  We considered the following two possible order
structures:  (i)  {\it Strong Structure} or the case in which the
%
\vskip 3mm


\noindent BIBLIOGRAPHY
\vskip 3mm

\noindent Bradley, R. A., Katti, S. K.,  and Coons, I. J. (1962). Optimal
scaling for ordered categories. {\it Psychometrika}, {\bf 27}, 355--374.

\vskip 3mm

\noindent  Goodman, L. A. (1979). Simple models for the analysis of
associations in cross classified data having ordered categories.
{\it J. Amer. Statist. Assoc.},  {\bf 74}, 537--552.

\vskip 3mm

\noindent Greenacre, M. J. (1984). {\it Correspondence Analysis}.  New York: Academic Press.
\vskip 3mm



\noindent  Haberman, S. J. (1974). Log-linear models for frequency tables
with ordered classification. {\it Biometrics},  {\bf 30}, 589--600.
\vskip 1in




**References should be cited in the text by author and date.  Please list the references in alphabetical order.
\vskip 4mm


**Equations and tables may be centered.
\vskip 4mm


**When sending {\TeX} or {\LaTeX} files, please be sure to include all the
input and style files. It is recommended to put the article on a disk
and run it from a different computer and check for error messages before
sending to {\it CIS}.

\end{document}
