%%%%%%%%%%%%%%%%%%%%%%%%%%%%%%%%%%%%%%%%%
% Author: bmarron
% Origin: 07 Apr 2017
% Final:
%%%%%%%%%%%%%%%%%%%%%%%%%%%%%%%%%%%%%%%%%



%----------------------------------------------------------------------------------------
%	PACKAGES AND OTHER DOCUMENT CONFIGURATIONS
%----------------------------------------------------------------------------------------

\documentclass[10pt]{article}	%use

\usepackage{lipsum} % Package to generate dummy text throughout this template
\usepackage{csquotes}
\usepackage[natbibapa]{apacite}
\usepackage[english]{babel}
\usepackage{amsmath}
\usepackage{amsthm}
\usepackage{amssymb}
\usepackage{pdfpages}
\usepackage{verbatim}
\usepackage{bigfoot}
\usepackage{multirow}


\usepackage[sc]{mathpazo} % Use the Palatino font and the Pazo fonts for math
\usepackage[T1]{fontenc} % Use 8-bit encoding that has 256 glyphs
\linespread{1.05} % Line spacing - Palatino needs more space between lines
\usepackage{microtype} % Slightly tweak font spacing for aesthetics

\usepackage[margin=25mm, top=32mm]{geometry} % Document margins
\usepackage{multicol} % Used for the two-column layout of the document
\usepackage[hang, small,labelfont=bf,up,textfont=it,up]{caption} % Custom captions under/above floats in tables or figures
\usepackage{booktabs} % Horizontal rules in tables
\usepackage{float} % Required for tables and figures in the multi-column environment - they need to be placed in specific locations with the [H] (e.g. \begin{table}[H])
%\usepackage{hyperref} % Interferes with cite.sty 

%\usepackage{lettrine} % The lettrine is the first enlarged letter at the beginning of the text
\usepackage{paralist} % Used for the compactitem environment which makes bullet points with less space between them

\usepackage{abstract} % Allows abstract customization
\renewcommand{\abstractnamefont}{\normalfont\bfseries} % Set the "Abstract" text to bold
\renewcommand{\abstracttextfont}{\normalfont\small\itshape} % Set the abstract itself to small italic text

\usepackage{titlesec} % Allows customization of titles
\renewcommand\thesection{\Roman{section}} % Roman numerals for the sections
\renewcommand\thesubsection{\arabic{subsection}} % Arabic numerals for subsections
\titleformat{\section}[block]{\large\scshape\centering}{\thesection.}{1em}{} % Change the look of the section titles
\titleformat{\subsection}[block]{\large}{\thesubsection.}{1em}{} % Change the look of the section titles

\usepackage{fancyhdr} % Headers and footers
\pagestyle{fancy} % All pages have headers and footers
\fancyhead{} % Blank out the default header
\fancyfoot{} % Blank out the default footer
\fancyhead[R]{\date{\today}}
\fancyfoot[C]{\thepage} % Custom footer text
%\fancyhead[C]{Running title $\bullet$ November 2012 $\bullet$ Vol. XXI, No. 1} % Custom header text
%\fancyfoot[RO,LE]{\thepage} % Custom footer text

%\usepackage{everypage} % Required for watermarks
%\usepackage{draftwatermark}
%\SetWatermarkLightness{0.95}
%\SetWatermarkScale{1}

%-------------------------------------------------------------
% NEW COMMANDS
%---------------------------------------------------------------------
%This command creates a box marked ``To Do'' around text.
%To use type \todo{  insert text here  }.
\newcommand{\todo}[1]{\vspace{5 mm}\par \noindent
\marginpar{\textsc{To Do}}
\framebox{\begin{minipage}[c]{0.95 \textwidth}
\tt\begin{center} #1 \end{center}\end{minipage}}\vspace{5 mm}\par}


\newcommand{\sun}{\ensuremath{\odot}} % sun symbol is \sun


% To use paragraph indents
%\begin{myindentpar}{2em}
% blah blah blah
%\end {myindentpar}
\newenvironment{myindentpar}[1]%
   {\begin{list}{}%
       {\setlength{\leftmargin}{#1}}%
           \item[]%
   }
     {\end{list}}

%----------------------------------------------------------------------------------------
%	TITLE SECTION
%----------------------------------------------------------------------------------------

\title{\vspace{-15mm}\fontsize{14pt}{10pt}\selectfont\textbf{Personal Statement}} % Article title

\author{
\large
\textsc{Bruce D. Marron} \\ %\thanks{A thank you or further information}\\[2mm] % Your name
%\normalsize Portland State University \\ % Your institution
\vspace{-5mm}
}
\date{}

%----------------------------------------------------------------------------------------
\begin{document}
\maketitle                % Insert title
\thispagestyle{fancy}     % All pages have headers and footers

With the career goal of public service through teaching and research, my personal goal is to work as an \enquote{honest, effective expert}\footnote{Ehrenfeld,D. 1993. \enquote{Down from the Pedestal--A New Role for Experts} in Beginning again: People and nature in the new millennium. Oxford University Press.}  providing other agricultural researchers and decision makers (agriculturists, land use planners, and agricultural policy makers) with scenario-based, simulation modeling tools and probability-based inference tools for evaluating the ecological robustness, resiliency, stability, and adaptability of agricultural-ecological landscapes. My intention is to hasten and ensure a successful, global transition to sustainable food production systems by defining and then modeling the set of complex linkages and relationships that would provide for mutually beneficial transfers of matter, energy, and information between the agriculturally-organized and the naturally-organized components of a multi-functional landscape. That is, my research aims at investigating the necessary and sufficient conditions for generating agricultural-ecological mutualism at the regional (landscape) scale over time. Such investigations require the evaluation of the dynamic, landscape-scale design patterns that are created by the confluence of (a) whole-system, natural resource management policies, (b) spatially-explicit ecological processes, (c) natural and agricultural disturbance regimes, and perhaps most importantly, (d) principles and practices of contemporary, soil biota-based, agroecology.\\

The necessary and sufficient conditions for realizing agricultural-ecological mutualism on any given landscape are far from certain. And it remains unclear how to spatially and temporally integrate agriculture into the natural world at the landscape scale so that the benefits of agricultural-ecological mutualism could be realized. There are many reasons why this is so. For example, landscapes are spatially and temporally dynamic over multiple scales of both. The resultant heterogeneity consists of a panoply of scale-dependent and perception-dependent mosaics that result from the complex interactions of biological, physical, and social forces and fluxes. Unraveling the mechanisms responsible for the relationships between landscape patterns and ecological processes is difficult not only because of the complexities of interaction, but because experimentation and hypothesis testing at such broad spatial scales must be done with models when extrapolation from small-scale experimentation is invalid. Ultimately, having agricultural components on a landscape adds additional complexities because of their inputs of matter, energy, and genetic information to the coupled, human-natural system. \\

From my perspective, agricultural-ecological mutualism, if realized, would be a landscape-scale emergent property that offers the prospect of generating virtuous cycles to the benefit of both people and the biosphere. More specifically, agricultural-ecological mutualism would (a) maintain high, annual net primary productivity (ANPP) in both agricultural and ecological (natural) components, (b) enhance the robustness and resilience of both components, (c) enhance and maintain soil fertility, and (d) maintain necessary and sufficient biodiversity as grist for the global evolutionary mill. Clearly, high ANPP implies closed nutrient and energy loops that trap greater amounts energetic and material flows across a landscape thereby creating eco-functional redundancies and building variety in system-level regulators via food web expansion. Enhanced robustness and resilience implies diverse assemblages of aboveground flora and below ground soil biota that can drive the ecological processes and dynamics required for maximal energy capture. And necessary biodiversity implies that agricultural-ecological mutualism boosts a landacape's evolutionary potential so that evolution (an information processing learning algorithm) can rapidly explore the enormous state space of possible patterns of biological form and function in order to provide adaptations (biologically \enquote{fit} organisms) in our epoch of dramatic global climate change and mass extinction. \\


Beyond the academic, I have a strong affinity for the arts and a very real commitment to social and planetary justice. I have spent time as a semi-professional musician and practice movement and martial arts. I have provided \textit{pro bono} consulting services to the United Way and have taught Latino kids in a program sponsored by Catholic Charities. Having grown up in Ojo Caliente, New Mexico and traveled substantially in Mexico, I relish Chicano and Native American cultures and do my best to give back to these communities. And finally, my family and I have accepted our responsibility for reducing our negative environmental impacts by relentlessly seeking and making ever simpler lifestyle choices. 




%----------------------------------------------------------------------------------------
%	REFERENCE LIST
%----------------------------------------------------------------------------------------
%\newpage
%\bibliographystyle{/usr/local/share/texmf/tex/latex/apacite/apacite}
%\bibliography{/home/bmarron/Desktop/BibTex/My_Library_20170125}


\end{document}
