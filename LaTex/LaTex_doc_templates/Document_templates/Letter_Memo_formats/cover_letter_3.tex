% modify, as needed
% $ sudo mousepad /usr/local/share/texmf/tex/latex/moderncv/moderncv.cls
% 
%
%%%%%%%%%%%%%%%%%%%%%%%%%%%%%%%


\documentclass[11 pt,letterpaper]{moderncv}        % possible options include font size ('10pt', '11pt' and '12pt'), paper size
                                                   % ('a4paper', 'letterpaper', 'a5paper', 'legalpaper', 'executivepaper' 'landscape')
                                                   % and font family ('sans' 'roman')

\usepackage{newcent}

                                                   % moderncv themes
\moderncvstyle{classic}                            % style options are 'casual' 'classic', 'oldstyle' 'banking'
\moderncvcolor{grey}                              % color options 'blue' 'orange', 'green', 'red', 'purple', 'grey' 'black'
\renewcommand{\familydefault}{\rmdefault}         % to set the default font; use '\sfdefault' for the default sans serif font,
                                                    % '\rmdefault' for the default roman one, or any tex font name
%\nopagenumbers{}                                  % uncomment to suppress automatic page numbering for CVs longer than one page

                                                  % character encoding
\usepackage[utf8]{inputenc}                       % if you are not using xelatex ou lualatex, replace by the encoding you are using
\usepackage[T1]{fontenc}                           % Use 8-bit encoding that has 256 glyphs
\usepackage{ragged2e}
\usepackage{csquotes}

                                                   % adjust the page margins
%\usepackage[scale=0.915]{geometry}
\usepackage[left=0.5in, right=0.5in, top=0.5in, bottom=0.5in ]{geometry}


%\setlength{\hintscolumnwidth}{3cm}                % if you want to change the width of the column with the dates
%\setlength{\makecvtitlenamewidth}{10cm}           % for the 'classic' style, if you want to force the width allocated to your name and
                                                   % avoid line breaks. be careful though, the length is normally calculated to avoid
                                                   % any overlap with your personal info; use this at your own typographical risks...


                                                    % to show numerical labels in the bibliography (default is to show no labels); 
                                                    % only useful if you make citations in your resume
%\makeatletter
%\renewcommand*{\bibliographyitemlabel}{\@biblabel{\arabic{enumiv}}}
%\makeatother
%\renewcommand*{\bibliographyitemlabel}{[\arabic{enumiv}]}% CONSIDER REPLACING THE ABOVE BY THIS

                                                    % bibliography with mutiple entries
%\usepackage{multibib}
%\newcites{book,misc}{{Books},{Others}}

%----------------------------------------------------------------------------------
%            Contact Info
% ------------------------------------------------------------------------------------
                                                    % personal data
\name{Bruce D.}{Marron, M.S.}
%\title{Resumé title}                               % optional, remove / comment the line if not wanted
\address{1644 SE 40th Ave}{Portland, OR 97214}{}      % optional, remove / comment the line if not wanted; the "postcode city" and
                                                      % "country" arguments can be omitted or provided empty
\phone[mobile]{(503)~890~5871}                        % optional, remove / comment the line if not wanted
%\phone[fixed]{(503)~236~8483}                        % optional, remove / comment the line if not wanted
%\phone[fax]{+3~(456)~789~012}                         % optional, remove / comment the line if not wanted
\email{soil.foodwebs@gmail.com}                               % optional, remove / comment the line if not wanted
\homepage{www.linkedin.com/in/bruce-marron-05496191}                         % optional, remove / comment the line if not wanted
%\extrainfo{}                                          % optional, remove / comment the line if not wanted
%\photo[64pt][0.4pt]{picture}                       % optional, remove / comment the line if not wanted; '64pt' is the height the picture
                                                   % must be resized to, 0.4pt is the thickness of the frame around it (put it to 0pt
                                                   % for no frame) and 'picture' is the name of the picture file
%\quote{Some quote}                                 % optional, remove / comment the line if not wanted

%----------------------------------------------------------------------------------
%          Begin Document
% ----------------------------------------------------------------------------------
\begin{document}

%---------------------------------------------------
%        Recipient Data
%---------------------------------------------------
\recipient{Portland Bureau of Emergency Management}{
9911 SE Bush St.\\
Portland, OR 97266}
\date{October 30, 2017}
\opening{\begin{center}
\vspace{-1.95 cm}
%\small{\textbf{RE: Position \#2017-00775, Neighborhood Emergency Program Specialist}}
\end{center}
Dear Bureau of Emergency Management,}
\closing{}
%\enclosure[Attached]{curriculum vit\ae{}}          % use an optional argument to use a string other than "Enclosure", or redefine \enclname

% ---------------------------------------------------
%       Begin Letter
%----------------------------------------------------

\makelettertitle
\justifying
Thank you for the opportunity to apply for the position of Neighborhood Emergency Team Program Specialist (Job \#2017-00775). I am bilingual in Spanish. I am educationally prepared for this position; I have two Master’s degrees, have passed comprehensive exams in a doctoral program at Portland State University (PSU), and have Graduate Certificates in both Statistics and in Sustainability, also from PSU. Additionally, I have experience as a volunteer firefighter, a decade of experience in education, a strong history of written and oral presentations to diverse audiences, prior experience in volunteer recruitment and leadership, a proven record of program and resource management and, as an 18 year resident of Portland, I have a deep understanding of and connection to our city. 

Hurricane Harvey, California wildfires, earthquakes in Mexico, and mass shootings are stark reminders that we are all vulnerable to natural and man-made disasters. The Portland Bureau of Emergency Management (PBEM) has done an outstanding job in building a corps of volunteer emergency responders through its Portland Neighborhood Emergency Team (NET) program. Ensuring the vibrancy, efficacy, and relevance of the Portland NET program into the future rests on three key tasks: (1) the continued successful organization, training, and support for current volunteers, (2) the successful integration of the NET program into every Portland neighborhood recognizing that different neighborhoods have different challenges, and (3) recruitment of new NET volunteers from those with diverse backgrounds, and those who are underserved and who have special needs. As a culturally aware, creative, warm-hearted, enthusiastic person with advanced scientific training, advanced scholarship, reliable multi-tasking experience, and excellent presentation skills, I am uniquely and ideally qualified to take on the challenges of these three key tasks. 

I have demonstrated ability to recruit and coordinate volunteers, handle logistics, coordinate scheduling, and manage tasks; and to safeguard equipment. As a former Portland Public Schools (PPS) science teacher I have nearly 10 years of experience recruiting and managing both parent and student volunteers. I have two years experience as a Harvest Leader with Portland Fruit Tree Project handling logistics and safely managing adult volunteers. Some harvest events were conducted in Spanish. I have over nine years of experience in project development and management as a project manager, as a consultant and independent contractor, as a program development specialist, as a curriculum development specialist, as a QAQC specialist, and as a regulatory compliance specialist. Additionally, I have years of experience as an environmental analytical chemist and chemistry science teacher in handling and maintaining extensive inventories of equipment and supplies.
 
I have excellent observational, analytical, and logical skills and the ability to reach sound and logical fact-based conclusions. I received a National Science Foundation Integrative Graduate Education and Research Traineeship (IGERT) Fellowship which furthered my training in the analysis of complex systems, statistics, and interdisciplinary collaboration. With advanced training in data science and statistical analysis, I am capable of designing tailor-made survey instruments and data collection tools to gather critical information for analysis of outcomes, and to make sound judgments from the data. Additionally, I have demonstrated ability to set, track, and evaluate goals when working with underserved communities. As a teacher at Kellogg Middle School (a Title I School) and for the Puentes program, I regularly set, tracked, and evaluated individual student performance goals as well as school-wide goals for traditionally underserved communities.

I have extensive public speaking and presentation experience including over 20 presentations to the Richmond Neighborhood Association as the Sustainability Committee chair. I have given dozens of project presentations to audiences as diverse as the university scientific community; the City of Portland (pitching my \textit{Victory Garden Plus!} program); US International Association of Landscape Ecologists conference attendees; EcoDistricts Summit conference attendees; children, young adults (grades 6-12) and their parents; and adults with special needs at On-the-Move Community Integration. I thoroughly enjoy public presentations!

I understand the principles and practices of public outreach, volunteer recruitment, and volunteer retention. Community presence, engaging and memorable events, and direct personal connections and communication are essential. The four components common to all successful outreach campaigns (awareness building, curiosity satisfaction, commitment fostering, and data-driven feedback) certainly apply to NET recruitment and growth. But also I believe that successful volunteer outreach and retention must be creative and community-oriented. For example, Portland NET awareness building could include creative eye-catchers such as biking en masse to a boom box beat at a Sunday Parkways event. Curiosity satisfaction could be met by making regular, short and snappy pitches on KBOO radio shows that serve our many wonderfully diverse communities. And, taking a tip from Kansas City, how about a NETs rodeo event to enhance volunteer retention? Volunteer retention is positively correlated with realistic engagement activities. Of course, I would use data tracking and statistical analysis to evaluate the success of all outreach efforts and make recommendations for future events based on careful analysis of evidence-based metrics.

I have excellent oral and written communication skills. Working as a project manager, a doctoral student and IGERT fellow, a consultant, and a teacher, I have provided effective written and oral communication to community stakeholders, to clients, and to the university community in the context of final deliverables, project reports, information sharing documents, and instructional materials. I was contracted by the Oregon Environmental Council to write a white paper on reactive nitrogen in Oregon. I have been an editor for the Complexity Explorer Project, and I have served as a peer reviewer for many of my colleagues. I am highly computer literate (open-source software advocate and simulation modeler) and having worked under rigorous QAQC protocols (NQA-1), I have excellent record keeping and organizational skills. 

On a personal note, I know and love our city. Since 1999, I have invested in Portland’s well-being, vibrancy, and resiliency as a home owner, a teacher, a permaculturist, a musician, and a scientist. As a former volunteer fireman, and as a martial artist, I know very well the value of training and preparation. As a former performing artist, I recognize the value of creative, awareness-building events. With contacts throughout PPS, I am ideally positioned to recruit more young people into the Portland NET program. Having supported On-the-Move Community Integration for five years, I am excited and knowledgeable about inclusion of individuals with disabilities. I am accustomed to public scrutiny and accountability. I have a client-based, customer service attitude and am fully committed to public service. I stand ready and able to serve the City of Portland, the Portland Bureau of Emergency Management, and the Portland Neighborhood Emergency Team program. 

I look forward to speaking with you and would welcome requests for additional information. 

\noindent With best wishes, \\
\includegraphics[scale=.6]{graphics/signature.jpeg}\\
Bruce D. Marron,  M.S.\\


%\makeletterclosing

\end{document}


%% end of file `template.tex'.
