%%%%%%%%%%%%%%%%%%%%%%%%%%%%%%%%%%%%%%%%%
% The Legrand Orange Book
% LaTeX Template
% Version 1.4 (12/4/14)
%
% This template has been downloaded from:
% http://www.LaTeXTemplates.com
%
% Original author:
% Mathias Legrand (legrand.mathias@gmail.com)
%
% License:
% CC BY-NC-SA 3.0 (http://creativecommons.org/licenses/by-nc-sa/3.0/)
%
% Compiling this template:
% This template uses biber for its bibliography and makeindex for its index.
% When you first open the template, compile it from the command line with the 
% commands below to make sure your LaTeX distribution is configured correctly:
%
% 1) pdflatex main
% 2) makeindex main.idx -s StyleInd.ist
% 3) biber main
% 4) pdflatex main x 2
%
% After this, when you wish to update the bibliography/index use the appropriate
% command above and make sure to compile with pdflatex several times 
% afterwards to propagate your changes to the document.
%
% This template also uses a number of packages which may need to be
% updated to the newest versions for the template to compile. It is strongly
% recommended you update your LaTeX distribution if you have any
% compilation errors.
%
% Important note:
% Chapter heading images should have a 2:1 width:height ratio,
% e.g. 920px width and 460px height.
%
%%%%%%%%%%%%%%%%%%%%%%%%%%%%%%%%%%%%%%%%%

%----------------------------------------------------------------------------------------
%	PACKAGES AND OTHER DOCUMENT CONFIGURATIONS
%----------------------------------------------------------------------------------------

\documentclass[11pt,fleqn]{book} % Default font size and left-justified equations

\usepackage[top=3cm,bottom=3cm,left=3.2cm,right=3.2cm,headsep=10pt,a4paper]{geometry} % Page margins

\usepackage{xcolor} % Required for specifying colors by name
\definecolor{ocre}{RGB}{243,102,25} % Define the orange color used for highlighting throughout the book

% Font Settings
\usepackage{avant} % Use the Avantgarde font for headings
%\usepackage{times} % Use the Times font for headings
%\usepackage{newcent} % Use the Times font for headings
\usepackage{mathptmx} % Use the Adobe Times Roman as the default text font together with math symbols from the Sym­bol, Chancery and Com­puter Modern fonts

\usepackage{microtype} % Slightly tweak font spacing for aesthetics
\usepackage[utf8]{inputenc} % Required for including letters with accents
\usepackage[T1]{fontenc} % Use 8-bit encoding that has 256 glyphs

% Bibliography
\usepackage[style=apa,
            sorting=nyt,
            sortcites=true,
            autopunct=true,
            hyperref=true,
            abbreviate=false,
            backref=true,
            doi=false,
            url=false,
            backend=biber]{biblatex}


%\usepackage[backref=true,                % 
%            hyperref=true,               % 
%            firstinits=true,             %
%            indexing=true,               %
%            url=false,                   % 
%            style=alphabetic,            %  style=debug, alphabetic
%            backend=biber,               % 
%            doi=false,
%            texencoding=utf8,
%            bibencoding=utf8]{biblatex}
\addbibresource{bibliography2.bib} % BibTeX bibliography file

%\defbibheading{bibempty}{}
\usepackage[babel]{csquotes}
\usepackage[american]{babel}
\DeclareLanguageMapping{american}{american-apa}

% Index
\usepackage{calc} % For simpler calculation - used for spacing the index letter headings correctly
\usepackage{makeidx} % Required to make an index
\makeindex % Tells LaTeX to create the files required for indexing
\usepackage{verbatim}

%----------------------------------------------------------------------------------------

\input{structure} % Insert the commands.tex file which contains the majority of the structure behind the template

\begin{document}

\let\cleardoublepage\clearpage

%----------------------------------------------------------------------------------------
%	TITLE PAGE
%----------------------------------------------------------------------------------------

\begingroup
\thispagestyle{empty}
\AddToShipoutPicture*{\put(6,5){\includegraphics[scale=1]{portal-energy-background}}} % Image background

\centering
\vspace*{9cm}
\par\normalfont\fontsize{35}{35}\sffamily\selectfont
Etude et estimation de la consommation totale d'électricité par calage avec ou sans réduction du nombre de variables auxiliaires.\par % Book title
\vspace*{1cm}
{\Huge Amin El Gareh et Cheikh Med Lmami Bezeid}\par % Author name
\endgroup

%---------------------------------------------------


\chapterimage{background.pdf} % Chapter heading image


\renewcommand\contentsname{Table des Matières}
\tableofcontents

%----------------------------------------------------------------------------------------
%	CHAPTER 1
%----------------------------------------------------------------------------------------


\chapterimage{chapter_head_1.pdf} % Chapter heading image


\chapter{Présentation des données d'électricité}

\section{Introduction des données}

On s’intéresse à des données d'électricité irlandaise de très grandes dimensions. Il s'agit de la consommation d'électricité enregistrée toutes les 30 minutes pendant 2 semaines (du lundi 5 octobre 2009 à 00:00 au dimanche 18 octobre 2009 à 23h30) pour 6291 individus: résidentiels, petites \& moyennes entreprises, et autres \parencite{garnett}.


%------------------------------------------------

\section{Etude descriptive des données}

Notre objectif ici est d'isoler le ou les comportements de consommation d'un ensemble d'individus appartenant à une même classe: résidentiels, petites \& moyennes entreprises, ou autres. 

\subsection{Consommation moyenne}
On considère la consommation moyenne comme étant la consommation totale prise en moyenne sur toute la population et par jour de la semaine. Sur la figure 1.1, on a représenté cette consommation moyenne en fonction du temps en minutes, et les jours de la semaine y sont délimités par des traits verticals rouges. 
L'analyse de la courbe des individus "résidentiels" et "autres", nous révèle le caractère cyclique de la consommation moyenne sur une période de 24h. La courbe des individus "petites \& moyennes entreprises" indique une consommation moyenne qui s'apparente être cyclique entre le lundi et le vendredi, mais qui ne l'est pas le week-end.




% ----------------------------------------------------------------------------------------
% 	BIBLIOGRAPHY
% ----------------------------------------------------------------------------------------

\printbibliography


%----------------------------------------------------------------------------------------

\end{document}
