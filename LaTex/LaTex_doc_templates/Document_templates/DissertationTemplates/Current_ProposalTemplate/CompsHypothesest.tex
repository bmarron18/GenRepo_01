This is Comps proposal sketch


\section{Research Hypotheses, Questions, and Objectives}
as defined below in terms of soil-based, agroecological mutualisms,

Objective--to develop a process-based, spatially explicit, dynamic model linking belowground soil food webs to aboveground productivity, and 2) 

be able to evaluate landscape-scale agroecological patterns can generate virtuous ecological processes for the sustainable production of food. 


there are ecological functionalities which link agricultural components of landscapes to natural components of landscapes to create multi-functional landscape wholes. Ecosystem goods and services can thus be conceptualized as emergent properties of multi-functional landscape systems. This suggests that, not only must specific component subsystems must be present, but that the activation of many subsystem processes are likely to be threshold controlled by one or more Heaviside-like step functions (i.e., thresholds exist that turn processes on and off). 



\begin{description}
	\item{\textit{\textbf{Hypothesis 1} -- The agricultural practices of the Lowland Maya during their Classical Period imposed a disturbance regime on the Yucat\'{a}n Peninsula that created a robust, resilient, and adaptive successionally developed agroecosystem; a verifiable example of agricultural-ecological mutualism.}}
	
	\underline{Important hypothesis driven questions}
	\begin{description}
		\item{\textit{Question 1.1} -- What was the frequency, intensity, and extent of the Mayan Classical Period agricultural disturbance regime?}
		\item{\textit{Question 1.2} -- What were the estimated caloric needs of the human population met by annual agricultural production during the Classical Period?}
		\item{\textit{Question 1.3} -- What was the estimated ANPP of the Mayan successionally developed agroecosystem during the Classical Period?}
		\item{\textit{Question 1.4} -- What were the major successional stages of the Mayan successionally developed agroecosystem?}
		\item{\textit{Question 1.5} -- What are the key spatial, temporal, ecological, and biological state variables that define the major successional stages of the Mayan successionally developed agroecosystem?}
		\item{\textit{Question 1.6} --  What are the density distributions of the agricultural-ecological mutualism state variables for the Mayan successionally developed agroecosystem?}
		\item{\textit{Question 1.7} -- What is the relationship between soil food webs and the major successional stages of the Mayan successionally developed agroecosystem?}
	\end{description}
	
	\underline{Important hypothesis directed objectives}
	\begin{description}
	\item{\textit{Objective 1.1} -- Develop a detailed definition of, and rationale for, agricultural-ecological mutualism.}
	\item{\textit{Objective 1.2} -- Define the necessary and sufficient spatial, temporal, ecological, and biological state variables for agricultural-ecological mutualism using historical and modern data from the Mayan successionally developed agroecosystem.}
	\item{\textit{Objective 1.3} -- Compile distribution data for each of the state variables identified in Objective 1.2 using the Mayan successionally developed agroecosystem as a model.}
	\item{\textit{Objective 1.4} -- Evaluate the level of agricultural-ecological mutualism for a variety of non-Mayan agricultural practices where long-term ecological and agricultural datasets are available.}	
	\end{description}
	
	
	\item{\textit{\textbf{Hypothesis 2} -- Maximizing the benefits of agricultural-ecological mutualism for any given landscape requires tuning the frequencies of spatio-temporal patterns of an agricultural successional mosaic.}}
	
	\underline{Important hypothesis driven questions}
	\begin{description}
		\item{\textit{Question 2.1} -- What is the relationship between agricultural-ecological mutualism and agricultural disturbance regimes based on the Mayan successionally developed agroecosystem model?}		
		\item{\textit{Question 2.2} -- Is it possible to develop a simple taxonomy of soil food webs classified based on morphological, semi-quantitative assessments of soil biota done \textit{in situ}?}
		\item{\textit{Question 2.3} -- Is it possible to correlate ANPP, nutrient cycling, and the rate of long-term soil carbon deposition (humic and fulvic acids) with taxonomic classes of soil food webs?}
		\item{\textit{Question 2.4} -- For ecological-agricultural mutualism to be applied to a given landscape, how do the historic disturbance regimes and successional patterns affect the appropriate state variables of the landscape?}
		\item{\textit{Question 2.5} -- For ecological-agricultural mutualism to be applied to a given landscape, what values of the agricultural-ecological mutualism state variables would provide maximal agricultural and ecological productivity?}
	\end{description}
	
	\underline{Important hypothesis directed objectives}
	\begin{description}
	\item{\textit{Objective 2.1} -- Define the floral species density distributions per major successional stage of the Mayan successionally developed agroecosystem.}
	\item{\textit{Objective 2.2} -- Qualitatively assess soil biota (bacteria, fungi, protozoa, nematodes) for the major successional stages of the Mayan successionally developed agroecosystem.}
	\item{\textit{Objective 2.3} -- Correlate soil food web classes to successional stages using literature-derived data and field sample data.}
	\item{\textit{Objective 2.4} -- Build a LANDIS-II model (Biomass extension) of the Mayan successionally developed agroecosystem.}
	\end{description}
	
	
	\item{\textit{\textbf{Hypothesis 3} -- The assemblages of soil biota with the greatest benefit for agricultural-ecological mutualism are evolved through the cycles of disturbance-succession created by a successionally developed agroecosystem.}}
	
	\underline{Important hypothesis driven questions}
	\begin{description}
		\item{\textit{Question 3.1} -- To what extent are landscape successional patterns the result of disturbance-triggered dynamic changes in the assemblages of soil biota?}
		\item{\textit{Question 3.2} -- Which landscape patterns generate the functional soil food webs that are necessary and sufficient for agricultural-ecological mutualism?}
		\item{\textit{Question 3.3} -- Is it possible to use the outputs from LANDIS-II simulations to predict soil organic carbon dynamics based on rhizodeposition and soil food web classes rather than solely on decomposition-based (anaerobic) carbon pools?}
		\item{\textit{Question 3.4} -- Is it possible to use the outputs from LANDIS-II simulations to predict successful patterns of agricultural-ecological mutualism?}
	\end{description}
	
	\underline{Important hypothesis directed objectives}
	\begin{description}
	\item{\textit{Objective 3.1} -- Build a series of agricultural management scenario-based LANDIS-II models that explore the relationships between landscape spatio-temporal patterns, agricultural disturbance regimes, soil food web classes, soil organic carbon dynamics, and succession.}
	\item{\textit{Objective 3.2} -- Design a set of LANDIS-II simulation experiments to test the effects of different agricultural disturbance regime patterns on agricultural-ecological mutualism processes.}	
	\end{description}

\end{description}

\section{Expected Research Methods}
Standard methods of experimental design \citep{montgomery_design_2013} and statistical analyses \citep{R_computing} will be used along with supplementary methods of inference that may include Bayesian and probability-based approaches \citep{hoff_first_2009, lunn_bugs_2013, jaynes_probability_2003}, bootstrap and rerandomization methods \citep {lunneborg_data_2000}, multivariate methods \citep{johnson_applied_2007}, time series methods \citep{shumway_time_2011}, and machine learning methods \citep{alpaydin_introduction_2014}.\\

A process-based successional simulation model, LANDIS-II, will be used \citep{scheller_design_2007} and various dynamical systems sub-models \citep{haefner_modeling_2005} or Markov sub-models \citep{fink_markov_2014} may be developed.\\

Soil samples from agriculturally-derived successional stages in the Yucat\'{a}n Peninsula will be collected. These field samples will be qualitatively assessed for soil food web classes with measurements likely to include soil pH, redox potential, total bacterial biomass, total fungal biomass, hyphal diameter, protozoan counts, and nematode counts.
