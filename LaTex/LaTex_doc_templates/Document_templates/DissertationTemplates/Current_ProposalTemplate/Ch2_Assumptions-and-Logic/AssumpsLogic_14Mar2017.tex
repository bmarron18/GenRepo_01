\todo{Section 1.1 from list to paragraph format; tidy up Section 1.2}


\section{Research Program Assumptions}
The proposed research program is built on assumptions logically alloyed from the conceptualizations and theoretical considerations from a variety of disciplines through a broad swath of peer-reviewed literature. This section provides a brief narrative of those assumptions and logical connections. The following primary assumptions are presented as synthetic statements and are derived, in part, from this set of references:\\

\noindent \citep{ingham_interactions_1985}, \citep{maeder_soil_2002}, \citep{polis_food_1996}, \citep{van_der_heijden_unseen_2008}, \citep{lavelle_faunal_1997-1}, \citep{kladivko_tillage_2001}, \citep{doran_soil_2000-1}, \citep{bongers_nematode_1999}, \citep{brussaard_soil_1998}, \citep{coleman_peds_2008}, \citep{sackett_linking_2010}, \citep{kardol_how_2010},\citep{van_der_putten_empirical_2009}, \citep{bardgett_belowground_2014}, \citep{scheller_forest_2004}, \citep{williams_simple_2010}, \citep{fath_ecological_2007}, \citep{harte_maximum_2011}, \citep{gianinazzi_impact_1994}, \citep{eigen_hypercycle_1979}, \citep{nicolis_exploring_1989}

\begin{itemize}
  \item There are ecological functionalities which link agricultural components of landscapes to natural components of landscapes to create multi-functional landscape wholes...
  
  \item Ecosystem goods and services can be conceptualized as emergent properties of multi-functional landscape systems...
  
  \item Specific component subsystems must be present and the activation of many subsystem processes are likely to be threshold controlled by one or more Heaviside-like step functions (i.e., thresholds exist that turn processes on and off)...

  \item Plants are "super-organisms" composed of the autotroph and its rhizospheric symbionts. There are massive feedback loops between the autotroph and its symbionts governed by large numbers of biological transducers...
  
  \item Rhizospheric symbionts are species-rich communities of fungi and bacteria that feed directly from plant root exudates. These fungi and bacteria form the mutualisms that are the basis for a soil's rhizospheric-based soil food web component...

  \item The rhizospheric-based soil food web component is a set of species-rich communities of fungi, bacteria, protozoa, and nematodes. The rhizospheric-based soil food web component is linked to the soil's detritus-based soil food web component, likewise a species-rich set of communities of fungi, bacteria, protozoa, and nematodes. Together, these two individual food web components form one, reticulate and mutualistic whole (the soil food web) with highly complex topology and population dynamics...
  
  \item Homogeneous trophic levels do not exist in soil food webs although temporal distance from photosynthetic carbon can define soil food web typologies: Rhizospheric-based communities are $10^{1} - 10^{3} \ s$ from photosynthetic carbon, detritus-based communities are $ > 10^{3} \ s$ from photosynthetic carbon.  
  
  \item Powered by photosynthetic carbon, soil food webs are complex adaptive systems maintained far from thermodynamic equilibrium. Soil food webs (and their plant allies) are Darwinian systems with virtually unlimited evolutionary potential. The evolutionary potential of soil food webs is the foundation for predicting the ecosystem scale effects of global change.
  
 \item Soil food webs drive nutrient recycling, soil formation, and soil fecundity. Plant health and productivity in natural systems depend on the diversity and viability of communities of (aerobic) soil microbes. Generally, soil biology drives soil chemistry, not the reverse.
 
  \item Soil food web processes directly affect patterns of landscape succession through variations in fungal/bacterial/protozoa/nematode ratios.
  
  \item The greater the structural and functional similarity of an agroecosystem to the natural ecosystems in its biogeographic region the greater the likelihood that the agroecosystem will be sustainable. 
  
  \item Sustainable agriculture is defined as ecological-agricultural mutualism -- the set of complex linkages, relationships, and practices that provide for mutually beneficial transfers of matter, energy, and information between the agriculturally-organized and the naturally-organized components of a multi-functional landscape. 
  
  \item Ecological-agricultural mutualism maintains biodiversity and evolutionary potential.
  
  \item Renewable, long-term extraction of food from the global biosphere for human consumption requires ecologically functional and climate change resilient agroecosystems at the regional scale \citep{barnosky_approaching_2012}.  Here, the term \enquote{agroecosystem} is taken as a food production system that is biologically diverse, adaptive, and robust as well as functionally (ecologically) integrated into the natural landscape.
  
  \item A transition to agroecologically-based food production systems, especially at the regional scale, is expected to provide improved food security as well as improved water quality \citep{godfray_food_2010, schmidhuber_global_2007} 
  
  \item Mayan agricultural practices during the Classical Period may have produced a natural experiment in agroecological mutualism that was, in fact, a direct result of an agriculturally-based, landscape disturbance regime...
  
  \item The explicit effects of the Mayan agricultural practices at the landscape scale were responsible for spatio-temporal patterns that, in turn, generated virtuous ecological processes that were of benefit to both people and regional ecosystems...

\end{itemize}



\section{Research Program Logic}
\textit{\underline{If} agriculture is a landscape disturbance phenomenon \underline{and} landscape disturbance regimes are critical to ecosystem health, \underline{then} landscape disturbance regimes affect the long-term delivery of ecosystem goods and services \underline{and} appropriately-realized agricultural practices (i.e., disturbance regimes) at the landscape scale may increase ecosystem robustness, resilience, and adaptive capacity.}\\

Variants of this type of reasoning have fermented for quite some time and currently have coalesced to foreshadow a real science of agricultural-ecological mutualism \citep{swift_biodiversity_2004, gliessman_agroecology:_2015}. \citet{swift_biodiversity_2004} provide deep insights into the possible structural and functional relationships between agriculture, ecosystem functioning, and ecosystems services at the landscape scale including simulation-testable hypotheses such as,
\begin{myindentpar}{2em}
\enquote{We hypothesise that the relationship between species richness and specific ecosystem services at the landscape scale may follow a relationship analogous with that of the Vitousek-Hooper model -- together of course with all the attendant qualifications. That is to say that ecosystem services at the landscape scale are optimised by a diversity of land uses, but the number [of land-use types] that are required for optimisation is relatively small} \citep{swift_biodiversity_2004}.
\end{myindentpar}
\citet{gliessman_agroecology:_2015} clearly explains the ecological characteristics of agroecosystems in relation to disturbance regimes and successional development.  Both \citet{swift_biodiversity_2004} and \citet{gliessman_agroecology:_2015} point to the ecological, economic, and social benefits of the successional mosaics created by traditional practices of tropical agriculture. An interesting side note that could provide a useful evolutionary theoretical underpinning for agricultural-ecological mutualism is Kauffman's idea that co-evolution is the coupling of the fitness landscapes of multiple species \citep{kauffman_origins_1993}. \citet{kauffman_origins_1993} points out that such landscape structures may be able to be \enquote{tuned} for the adaptive success of all species and if so, would provide the \textit{raison d'\^{e}tre} for the existence of ecosystems.\\

\textit{\underline{If} it is the energetic and biochemical transformations brokered by rhizospheric fungal/bacterial/protozoa assemblages that are primarily responsible for nutrient cycling \underline{and} rhizospheric fungal/bacterial/protozoa assemblages are soil food webs fueled by the root exudates of living plants, \underline{then} different natural successional stages have unique soil food webs \underline{and} outcomes of interactions in the rhizosphere ultimately affect plant and soil community dynamics at the ecosystem scale \underline{and} maximizing the nutrient retention and trophic energy capture capabilities of any given landscape requires a set of necessary and sufficient, successionally derived, soil food webs.}\\

Although Lorenz Hiltner recognized over 100 years ago that the density and activity of microorganisms increases in the vicinity of plant roots, only recently has evidence accumulated to suggest that, in fact, plant-microbe interactions may actually direct nutrient cycling processes as well as provide critical autotrophic support \citep{ingham_interactions_1985, bais_role_2006}. The idea that complex soil food webs and not decomposition networks are responsible for the bulk of plant growth promoting processes, nutrient cycling, and novel microbial ecological functions is causing a revolution in microbial ecology with exciting possibilities for both agricultural and ecological management at the landscape scale \citep{prosser_role_2007, smith_plant_2009, pedraza_plant_2009, de_vries_soil_2013, hedlund_trophic_2004, holtkamp_soil_2008, minoshima_soil_2007, rygiewicz_soil_2010, wardle_how_1999}. Complex soil food webs are fueled by rhizodeposition of plant root exudates whereas decomposition networks are fueled by detritus (litter). \\

\textit{\underline{If} the primary productivity, delivery of ecosystem goods and services, and ecosystem resiliency of a landscape are dependent on soil food web biodynamics \underline{and} there are as-yet unknown thresholds required for functional soil food webs \underline{and} soil food webs are a function of the co-evolved communities produced by landscape-scale spatial and temporal patterns, \underline{then} scenario-based simulation modeling could be used to explore basic functional-biodiversity rules such as the above-ground biodiversity necessary to generate robust and resilient soil food webs.}\\

Monica Turner, a founding member of the American school of landscape ecology, points out that, \enquote{Elucidating the relationship between landscape pattern and ecological processes is a primary goal of ecological research on landscapes} \citep{turner_landscape_1989}. Recent work strongly suggests that there are as-yet unknown quantitative links between soil food web microbial diversity, spatio-temporal patterns, and ecosystem processes \citep{rygiewicz_soil_2010, compant_stephane_plant_2010, prosser_role_2007}. These links most certainly have been overlooked in ecological nutrient cycle models like CENTURY and DAISY which only consider the soil organic matter biochemical transformations that result from detritus-generated C and N litter pools \citep{smith_evaluation_1997, de_vries_soil_2013, kirschbaum_modelling_2002, roose_mathematical_2008}. Interestingly, there appears to be a kind of activation threshold to the initiation of functional soil food webs that may be keyed to succession and hence, to disturbance regimes.  
