\todo{ Fill out all sections of this outline  }

\section{Background}

\section{Questions}
\begin{outline}
\1 What are the key spatial, temporal, ecological, and biological state variables that define the major successional stages of the Mayan successionally developed agroecosystem?
\1 What are the density distributions of the agricultural-ecological mutualism state variables for the Mayan successionally developed agroecosystem?
\1 What is the relationship between soil food webs and the major successional stages of the Mayan successionally developed agroecosystem?
\1 What is the relationship between agricultural-ecological mutualism and agricultural disturbance regimes based on the Mayan successionally developed agroecosystem model?
\end{outline}



\section{Hypotheses}
\begin{outline}
\1 The agricultural practices of the Lowland Maya during their Classical Period imposed a disturbance regime on the Yucat\'{a}n Peninsula that created a robust, resilient, and adaptive successionally developed agroecosystem; a verifiable example of agroecological mutualism


\end{outline}


\section{Aims and Objectives}
\begin{outline}
\1 Evaluate landscape-scale agroecological patterns can generate virtuous ecological processes for the sustainable production of food
\1 Build a series of agricultural management scenario-based LANDIS-II models that explore the relationships between landscape spatio-temporal patterns, agricultural disturbance regimes, soil food web classes, soil organic carbon dynamics, and succession.
\1 Design a set of LANDIS-II simulation experiments to test the effects of different agricultural disturbance regime patterns on agroecological mutualism processes
\1 Qualitatively assess soil biota (bacteria, fungi, protozoa, nematodes) for the major successional stages of the Mayan successionally developed agroecosystem
\1 Build a LANDIS-II model of the Mayan successionally developed agroecosystem.

\end{outline}



\section{Methods}
