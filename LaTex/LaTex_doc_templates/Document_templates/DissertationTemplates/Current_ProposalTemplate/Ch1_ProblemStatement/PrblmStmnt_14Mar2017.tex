\todo{fill in Section 1.2; careful not to go on and on!}

\section{Introduction}
Will it be possible to feed ourselves over the next 50 years without crashing the biosphere? This is a pressing and complex problem. Even without social and political considerations, a variety of physical and ecological feedback mechanisms are currently operating in coupled human and natural systems at both global and national scales that are expected to place severe constraints on the national and regional production of ecosystem goods and services as early as 2050 \citep{zhao_drought-induced_2010, hoegh-guldberg_impact_2010, eigenbrod_impact_2011}. Ultimately, the outcomes of climate change (including ocean acidification, sea level rise, drought, shifts in storm regimes) the demands of economic growth (including increased natural resource extraction, pollution, appropriation of natural land area), and the surge of a human population to 9 billion are the drivers fueling these detrimental feedback mechanisms. Of perhaps greatest concern among those ecosystem service outputs threatened with decline are the two fundamental to human existence; namely, freshwater and food \citep{dodds_human_2013, rogers_facing_2008, lobell_climate_2011, wada_global_2010, zhao_drought-induced_2010, schmidhuber_global_2007, tilman_forecasting_2001, tilman_agricultural_2002}. 

Over the coming decades, agricultural production systems, which are already facing relentless competition for land, water, and energy, will be expected to deliver increasing amounts of diverse, highly-nutritious foods (especially animal-derived products) to an increasingly affluent global population while simultaneously (a) reducing the environmental externalities of production, (b) reducing impacts to natural biodiversity, (c) boosting crop biodiversity, 4) improving the social justice of food access, 5) adapting to shifting climates, intense weather events, and higher carbon dioxide levels, and 6) remaining economically viable in a globalized economy. Optimizing under this suite of constraints will require more than simply maximizing short-term yields. It will require a tremendous shift in agricultural practices to agroecology-based agronomic design principles with the long-term goals of sustainable yield, increased soil fertility, increased soil organic carbon, increased aboveground and belowground biodiversity, and increased transpiration efficiency. 

The research program presented in this document uses the theoretical constructs and metrics of contemporary, soil biota-based, agroecology coupled with (a) spatially-explicit ecological processes and (b) natural disturbance regimes to develop scenario-based, simulation modeling tools and probability-based inference tools for evaluating the ecological robustness, resiliency, stability, and adaptability of various cropping system design patterns at the regional (landscape) scale over time. The aim of the proposed research program is to hasten the successful transition to sustainable food production systems and enhance regional biodiversity by (a) defining sustainable agriculture as a set of agroecological mutualisms brokered by functional soil foodweb dynamics, (b) using the scenario-based, simulation modeling tools and probability-based inference tools to design spatially-explicit agroecological patterns that can enhance and maintain soil fertility and agricultural productivity over time, and (c) providing academic researchers as well as decision makers (agriculturists, land use planners, and agricultural policy makers) with accessible, science-based tools for the ecological assessment of agroecological landscapes.

To be realized, sustainable agriculture must be defined by more than "bucket lists" of desiderata (see Appendix A). The proposed research program seeks to identify the necessary and sufficient ecological conditions to establish patterns of sustainable agriculture on a given landscape by linking aboveground productivity to the biodiversity of belowground  soil biota through the processes mediated by functional soil food webs. 


\section{The Realities of Modern Industrial Agriculture}
\begin{outline}
\noindent \0 Agriculture directly couples humans to natural systems.
\1 Systems and coupled systems are...
\1 The material, energetic, and informational fluxes of ag production are...

\0 Modern industrial agriculture negatively affects regional ecology and regional effects aggregate to disrupt global ecological functionalities.
\1 Modern industrial ag is defined as ...
\1 Industrial ag is a primary source of anthropogenic inputs into natural biogeochemical cycles at the regional scale ... (the main inputs? the biogeochemical cycles of concern? the mechanisms and effects of inputs?)
\1 Industrial ag effects are aggregated to the global scale ... (mechanisms? examples?) 
\1 Regional inputs affect ecologies at a distance from source ... (mechanisms? examples?)
\1 The current state of worldwide ecosystem functionality is ... (examples? expectations?)
\end{outline}



\section{A Compound Problem Statement}
A tripartite problem statement.

\subsection{Element 1}
\begin{sloppypar}
\noindent \texttt{The current model of industrial agriculture jeopardizes sustainability and the long-term delivery of ecosystem goods and services.}\\
\end{sloppypar}

Economic sustainability, whether 'strong' or 'weak', relies on natural capital as the source for both primary and manufactured goods. Any substantial reduction in natural capital therefore reduces economic opportunity and potentially points to a decline in future human welfare \citep{pearce_blueprint_2000}. If the endowment of natural capital required for ecosystem goods and services is taken as the aggregate of functional terrestrial and aquatic ecosystems across the globe, then industrial agricultural reduces natural capital through loss of topsoil, creation of oceanic hypoxic zones, damage to hydrological systems, disruption of soil foodwebs, destruction of natural habitats across multiple scales, production of greenhouse gases, and loss of genetic diversity \citep{gliessman_agroecology:_2015}. Collectively these are externalized costs that result from (a) failure to properly value ecosystem goods and services, (b) failure to account for the true costs of environmental degradation and natural capital depletion, and (c) failure of markets and policies to provide incentives for sustainable stewardship of natural capital \citep{pearce_blueprint_2000}. \\

Industrial agriculture likewise threatens environmental sustainability. Humans rely on a functional biosphere for life support and industrial agriculture's externalized costs are environmentally devastating at the global scale \citep{tilman_agricultural_2002, wolfe_crop_2000, ceballos_accelerated_2015}. To take but two of many quantitative examples, the anthropogenic sources of reactive nitrogen that are directly linked to agriculture (i.e., fertilizer production by N-fixation from the Haber-Bosch process and fertilizer production by N-fixation from cultivation) now contribute reactive nitrogen to the biosphere at nearly double the rate of terrestrial nitrogen fixation (i.e., at about $13 \ x \ 10^{12} \ mol \ yr^{-1}$) \citep{canfield_evolution_2010}. And the agricultural sector is the third largest contributor of greenhouse gases behind transportation and power generation \citep{gliessman_agroecology:_2015}. \\

Finally, industrial agriculture can disrupt social sustainability. Confronted with industrial agricultural practices, rural agricultural communities may experience a loss of local control over agricultural production with a resulting loss in place-based knowledge, human capital, and civic engagement \citep{beus_conventional_1990, oecd_well-being_2001}. Recently concerns have been raised over the social costs of inequities in the final distribution of benefits and burdens associated with the production of ecosystem goods and services \citep{berbes-blazquez_towards_2016}.

\subsection{Element 2}
\begin{sloppypar}
\noindent \texttt{The externalized costs of industrial agriculture, a human population likely to swell to over 9 billion by 2050, the severe losses expected in biodiversity, declining overall biospheric health, and climate change demand agroecological mutualism.} \\
\end{sloppypar}

Agrocological mutualism is defined as the set of complex linkages and relationships that provide for mutually beneficial transfers of matter, energy, and information between the agriculturally- (anthropogenically-) organized and the naturally-organized components of a multi-functional landscape. Ideally, agroecological mutualism would (a) maintain high, annual net primary productivity (ANPP) in both components, (b) enhance the robustness and resilience of both components, and perhaps most importantly (c) maintain the necessary biodiversity and adaptive capacity which serves as grist for the global evolutionary mill.  High ANPP implies closed nutrient and energy loops that trap greater amounts energetic and material flows across a landscape thereby creating eco-functional redundancies and building variety in system-level regulators via trophic level expansion \citep{ashby_introduction_1955}. Enhanced robustness and resilience implies diverse assemblages of aboveground flora and belowground soil biota that can drive the ecological processes and dynamics required for maximal energy capture. And necessary biodiversity implies that evolution (an information processing learning algorithm) can rapidly explore the enormous state space of possible patterns of biological form and function in order to provide adaptations (i.e., biologically "fit" organisms) in our epoch of dramatic global climate change and mass extinction \citep{ceballos_accelerated_2015}.\\

\subsection{Element 3}
\begin{sloppypar}
\noindent \texttt{The mechanisms and processes that define the necessary and sufficient conditions for agroecological mutualism are currently unknown. However, there is evidence from a past, \enquote*{natural experiment} in the Yucat\'{a}n Peninsula that agroecological mutualisms, and hence sustainable agriculture, is directly related to spatio-temporal patterns on the landscape. }\\
\end{sloppypar}

The realization of agroecological mutualism is implicit in the bucket lists of \enquote*{sustainable agriculture} (see Appendix A) and in the concepts and methodologies of agroecology \citep{gliessman_agroecology:_2015}. It remains unclear, however, how to best spatially and temporally integrate agriculture into the natural world at the landscape scale so that the benefits of such mutualism are realized. There are many reasons why this is so. For example, landscapes are spatially and temporally dynamic over multiple scales of both. The resultant heterogeneity consists of a panoply of scale-dependent and perception-dependent mosaics that result from the complex interactions of biological, physical, and social forces and fluxes \citep{turner_landscape_1989}. Unraveling the mechanisms responsible for the relationships between landscape patterns and ecological processes is difficult not only because of the complexities of interaction, but because experimentation and hypothesis testing at broad spatial scales must be done with models when extrapolation from small-scale experimentation is invalid. Ultimately, having agricultural components on a landscape add additional complexities because of their inputs of matter, energy, and genetic information to the system. \\

The dynamic, managed mosaic created by the traditional agricultural practices of the Lowland Mayan in the Yucat\'{a}n Peninsula during their Classical Period (300-900 AD) may be an example of an as-yet, unexplained natural experiment in agroecological mutualism. The Lowland Mayan people maintained an average population density of 100-200 $persons \ km^{-2}$ throughout the entire Yucat\'{a}n peninsula of Mexico for at least 600 years during their Classic Period. At their peak, the Maya were feeding several million people \citep{gomez-pompa_lowland_2003} and densities in ancient cities such as Chunchucmil (about 2500  $persons \ km^{-2}$ ) and Tikal (about 3800 $persons \ km^{-2}$) were even greater than densities in comparable cities in the region today \citep{dahlin_reconstructing_2005, faust_maya_2001}.  After much debate and revision, the Lowland Mayan population data are now accepted in the literature. This raises many questions related to sustainable agriculture such as,
\begin{itemize}
  \item What were the practices that enabled the Classic Period Maya to successfully feed themselves for centuries in a tropical region that is poorly suited to agriculture?
  \item How were high levels of biodiversity sustained under such intensive agricultural production? 
  \item Why was there no severe ecosystem damage resulting from high population densities until the first part of the ninth century? (Note that while many hold up the Mayan Collapse as a classic example of ecological disaster, \citet{gomez-pompa_lowland_2003} and \citet{diamond_collapse:_2006} point to drought-fueled, political unrest as the most likely causes.)
  \item What are the mechanisms, dynamics, processes, and state variables responsible for this phenomenon and critically, are these mechanisms generalizable and portable to non-tropical regions of agricultural production?
\end{itemize}
The answers to these questions appear to lie in the relationships between agricultural disturbance and the biodiversity of soil biota, especially of bacteria and fungi. 

  
