\todo{ Fill out all sections of this outline  }

\section{Background}
A history of dynamic modeling efforts for biological systems
\begin{outline}
 \1 Model typologies
 \2 Generative vs. Statistical
 \2 Causal vs. Inferential (Evidential)
 \2 Compartmental (DEs) vs Agent-Based
\end{outline}  
 
 \section {Model }
 
 
 
 
\section{Questions}
\begin{outline}
\1 To what extent are landscape successional patterns the result of disturbance-triggered dynamic changes in the assemblages of soil biota?
\1 Which landscape patterns generate the functional soil food webs that are necessary and sufficient for agroecological mutualism?
\1 Is it possible to develop a simple taxonomy of soil food webs classified based on morphological, semi-quantitative assessments of soil biota done \textit{in situ}?
\1 Is it possible to correlate ANPP, nutrient cycling, and the rate of long-term soil carbon deposition (humic and fulvic acids) with taxonomic classes of soil food webs?
\1 For agroecological mutualism to be applied to a given landscape, how do the historic disturbance regimes and successional patterns affect the appropriate state variables of the landscape?
\1 For ecological-agricultural mutualism to be applied to a given landscape, what values of the agroecological mutualism state variables would provide maximal agricultural and ecological productivity?
\end{outline}


\section{Hypotheses}
\begin{outline}[enumerate]
\1 Following this, we hypothesize that the relationship between microbial diversity and ecosystem
functioning is different for nutrient poor and nutrient rich ecosystems. We expect that microbial
communities from nutrient rich ecosystems are functionally more redundant compared with
microbial communities from nutrient poor ecosystems where microbes need specific adaptations
to obtain resources \citep{van_der_heijden_unseen_2008}
\1 The assemblages of soil biota with the greatest benefit for agroecological mutualism are evolved through the cycles of disturbance-succession created by a successionally-developed agroecosystem.
\end{outline}


\section{Aims and Objectives}
\begin{outline}[enumerate]
\1 Develop a detailed definition of, and rationale for, agroecological mutualism
\1 Define the necessary and sufficient spatial, temporal, ecological, and biological state variables for agroecological mutualism using historical and modern data from the Mayan successionally developed agroecosystem
\1 Compile distribution data for each of the state variables using the Mayan successionally developed agroecosystem as a model
\1 Evaluate the level of agroecological mutualism for a variety of non-Mayan agricultural practices where long-term ecological and agricultural datasets are available
\1 Correlate soil food web classes to successional stages using literature-derived data and field sample data
\end{outline}

\section{Methods}
