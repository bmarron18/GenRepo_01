% \iffalse meta-comment
%
% Copyright 1993-2015
% The LaTeX3 Project and any individual authors listed elsewhere
% in this file. 
% 
% This file is part of the LaTeX base system.
% -------------------------------------------
% 
% It may be distributed and/or modified under the
% conditions of the LaTeX Project Public License, either version 1.3c
% of this license or (at your option) any later version.
% The latest version of this license is in
%    http://www.latex-project.org/lppl.txt
% and version 1.3c or later is part of all distributions of LaTeX 
% version 2005/12/01 or later.
% 
% This file has the LPPL maintenance status "maintained".
% 
% The list of all files belonging to the LaTeX base distribution is
% given in the file `manifest.txt'. See also `legal.txt' for additional
% information.
% 
% The list of derived (unpacked) files belonging to the distribution 
% and covered by LPPL is defined by the unpacking scripts (with 
% extension .ins) which are part of the distribution.
% 
% \fi

% This document will typeset the LaTeX sources as a single document.
% This will produce quite a large file (roughly 555 pages) and may
% take a long time.
%
% Some notes on processing this document are contained at the end
% of this document, after \end{document}
% 
% DPC 1997/11/17
%
% First a special index style for makeindex
%

\begin{filecontents}{source2e.ist}
actual '='
quote '!'
level '>'
preamble
"\n \\begin{theindex} \n \\makeatletter\\scan@allowedfalse\n"
postamble
"\n\n \\end{theindex}\n"
item_x1   "\\efill \n \\subitem "
item_x2   "\\efill \n \\subsubitem "
delim_0   "\\pfill "
delim_1   "\\pfill "
delim_2   "\\pfill "
% The next lines will produce some warnings when
% running Makeindex as they try to cover two different
% versions of the program:
lethead_prefix   "{\\bfseries\\hfil "
lethead_suffix   "\\hfil}\\nopagebreak\n"
lethead_flag       1
heading_prefix   "{\\bfseries\\hfil "
heading_suffix   "\\hfil}\\nopagebreak\n"
headings_flag       1

% and just for source2e:
% Remove R so I is treated in sequence I J K not I II III
page_precedence "rnaA"
\end{filecontents}



\documentclass{ltxdoc}

\listfiles

% Do not index some TeX primitives, and some common plain TeX commands.

\DoNotIndex{\def,\long,\edef,\xdef,\gdef,\let,\global}
\DoNotIndex{\if,\ifnum,\ifdim,\ifcat,\ifmmode,\ifvmode,\ifhmode,%
            \iftrue,\iffalse,\ifvoid,\ifx,\ifeof,\ifcase,\else,\or,\fi}
\DoNotIndex{\box,\copy,\setbox,\unvbox,\unhbox,\hbox,%
            \vbox,\vtop,\vcenter}
\DoNotIndex{\@empty,\immediate,\write}
\DoNotIndex{\egroup,\bgroup,\expandafter,\begingroup,\endgroup}
\DoNotIndex{\divide,\advance,\multiply,\count,\dimen}
\DoNotIndex{\relax,\space,\string}
\DoNotIndex{\csname,\endcsname,\@spaces,\openin,\openout,%
            \closein,\closeout}
\DoNotIndex{\catcode,\endinput}
\DoNotIndex{\jobname,\message,\read,\the,\m@ne,\noexpand}
\DoNotIndex{\hsize,\vsize,\hskip,\vskip,\kern,\hfil,\hfill,\hss}
\DoNotIndex{\m@ne,\z@,\z@skip,\@ne,\tw@,\p@}
\DoNotIndex{\dp,\wd,\ht,\vss,\unskip}

% Set up the Index and Change History to use \part
\IndexPrologue{\part*{Index}%
                 \markboth{Index}{Index}%
                 \addcontentsline{toc}{part}{Index}%
                 The italic numbers denote the pages where the
                 corresponding entry is described,
                 numbers underlined point to the definition,
                 all others indicate the places where it is used.}

\GlossaryPrologue{\part*{Change History}%
%                Allow control names to be hyphenated here...
                 {\GlossaryParms\ttfamily\hyphenchar\font=`\-}%
                 \markboth{Change History}{Change History}%
                 \addcontentsline{toc}{part}{Change History}}

% The standard \changes command modified slightly to better cope with
% this multiple file document.
\makeatletter
\def\changes@#1#2#3{%
  \let\protect\@unexpandable@protect
  \edef\@tempa{\noexpand\glossary{#2\space\currentfile\space#1\levelchar
                                 \ifx\saved@macroname\@empty
                                   \space
                                   \actualchar
                                   \generalname
                                 \else
                                   \expandafter\@gobble
                                   \saved@macroname
                                   \actualchar
                                   \string\verb\quotechar*%
                                   \verbatimchar\saved@macroname
                                   \verbatimchar
                                 \fi
                                 :\levelchar #3}}%
  \@tempa\endgroup\@esphack}
\makeatother

% Produce a Change Log and (2 column) Index.
\RecordChanges
\CodelineIndex
\EnableCrossrefs
\setcounter{IndexColumns}{2}

% Needed for documentation in ltoutenc.dtx
\usepackage{textcomp}

\begin{document}
 \title{The \LaTeXe\ Sources}
 \author{%
  Johannes Braams\\
  David Carlisle\\
  Alan Jeffrey\\
  Leslie Lamport\\
  Frank Mittelbach\\
  Chris Rowley\\
  Rainer Sch\"opf}

% This command will be used to input the patch file
% if that file exists.
\newcommand{\includeltpatch}{%
  \def\currentfile{ltpatch.ltx}
  \part{ltpatch}
  {\let\ttfamily\relax
    \xdef\filekey{\filekey, \thepart={\ttfamily\currentfile}}}%
  Things we did wrong\ldots
  \IndexInput{ltpatch.ltx}}



% Get the date from ltvers.dtx
\makeatletter
\let\patchdate=\@empty
\begingroup
   \def\ProvidesFile#1\fmtversion#2{\date{#2}\endinput}
   % \iffalse meta-comment
%
% Copyright 1993-2015
% The LaTeX3 Project and any individual authors listed elsewhere
% in this file.
%
% This file is part of the LaTeX base system.
% -------------------------------------------
%
% It may be distributed and/or modified under the
% conditions of the LaTeX Project Public License, either version 1.3c
% of this license or (at your option) any later version.
% The latest version of this license is in
%    http://www.latex-project.org/lppl.txt
% and version 1.3c or later is part of all distributions of LaTeX
% version 2005/12/01 or later.
%
% This file has the LPPL maintenance status "maintained".
%
% The list of all files belonging to the LaTeX base distribution is
% given in the file `manifest.txt'. See also `legal.txt' for additional
% information.
%
% The list of derived (unpacked) files belonging to the distribution
% and covered by LPPL is defined by the unpacking scripts (with
% extension .ins) which are part of the distribution.
%
% \fi
%
% \iffalse
%%% From File: ltvers.dtx
%
%<*driver>
% \fi
\ProvidesFile{ltvers.dtx}
             [2015/08/23 v1.0v LaTeX Kernel (Version Info)]
% \iffalse
\documentclass{ltxdoc}
\GetFileInfo{ltvers.dtx}
\title{\filename}
\date{\filedate}
 \author{%
  Johannes Braams\and
  David Carlisle\and
  Alan Jeffrey\and
  Leslie Lamport\and
  Frank Mittelbach\and
  Chris Rowley\and
  Rainer Sch\"opf}
\begin{document}
 \MaintainedByLaTeXTeam{latex}
 \maketitle
 \DocInput{\filename}
\end{document}
%</driver>
% \fi
%
% \CheckSum{150}
%
% \section{Version Identification}
% First we identify the date and version number of this release of
% \LaTeX, and set |\everyjob| so that it is printed at the start of
% every \LaTeX\ run.
%
% \StopEventually{}
%
% \changes{v1.0g}{1996/11/28}
%     {Check for old format modified /2319}
% \changes{v1.0f}{1996/11/20}
%     {Check for old format modified /2319}
% \changes{v1.0e}{1995/05/12}
%     {Add autoload docstrip guards}
% \changes{v1.0e}{1995/05/12}
%     {Check for format older than 1 year}
% \changes{v1.0d}{1994/05/25}
%     {Remove PRELIMINARY TEST RELEASE from startup banner
%      (spring is here)}
% \changes{v1.0b}{1994/04/12}
%     {Have version info generated automatically.}
% \changes{v1.0a}{1994/03/04}
%     {Initial version, split from latex.dtx}
% \changes{v1.0r}{2015/02/21}{Removed autoload code}
% \changes{v1.0t}{2015/06/23}
%     {set \cs{patch@level} in ltvers rather than in ltfinal/ltpatch}
%
% \begin{macro}{\fmtname}
% \begin{macro}{\fmtversion}
% \begin{macro}{\patch@level}
%    \begin{macrocode}
%<*2ekernel>
\def\fmtname{LaTeX2e}
\edef\fmtversion
%</2ekernel>
%<latexrelease>\edef\latexreleaseversion
%<*2ekernel|latexrelease>
   {2015/10/01}
%</2ekernel|latexrelease>
%<*2ekernel>
\def\patch@level{2}
%    \end{macrocode}
% \end{macro}
% \end{macro}
% \end{macro}
%
% Check that the format being made is not too old.
% The error message complains about `more than 5 years'
% but in fact the error is not triggered until 65 months.
%
% This code is currently not activated as we don't know if we already
% got to the last official 2e version (due to staff shortage or due to
% a successor (think positive:-)).
% \changes{v1.0i}{2001/06/04}{Check for old format disabled}
% \changes{v1.0k}{2004/01/28}{Check for old format made 5 years (pr/3601)}
% \changes{v1.0l}{2009/09/24}{Stop checking for old format}
%    \begin{macrocode}
\iffalse
\def\reserved@a#1/#2/#3\@nil{%
  \count@\year
  \advance\count@-#1\relax
  \multiply\count@ by 12\relax
  \advance\count@\month
  \advance\count@-#2\relax}
\expandafter\reserved@a\fmtversion\@nil
%    \end{macrocode}
% |\count@| is now the age of this file in months. Take a generous
% definition of `year' so this message is not generated too often.
%    \begin{macrocode}
\ifnum\count@>65
  \typeout{^^J%
!!!!!!!!!!!!!!!!!!!!!!!!!!!!!!!!!!!!!!!!!!!!!!!!!!!!!!!!!!!!!!!!!!^^J%
!  You are attempting to make a LaTeX format from a source file^^J%
!  That is more than five years old.^^J%
!^^J%
!  If you enter <return> to scroll past this message then the format^^J%
!  will be built, but please consider obtaining newer source files^^J%
!  before continuing to build LaTeX.^^J%
!!!!!!!!!!!!!!!!!!!!!!!!!!!!!!!!!!!!!!!!!!!!!!!!!!!!!!!!!!!!!!!!!!^^J%
}
   \errhelp{To avoid this error message, obtain new LaTeX sources.}
   \errmessage{LaTeX source files more than 5 years old!}
\fi
\let\reserved@a\relax
\fi
%    \end{macrocode}
%
% \changes{v1.0p}{2015/01/22}{Preserve any \cs{everyjob} material inserted
%   by a loader (\texttt{.ini} file)}
% \changes{v1.0v}{2015/08/23}{Allow negative patchlevel for pre-release}
%    \begin{macrocode}
  \ifnum\patch@level=0
    \everyjob\expandafter{\the\everyjob
      \typeout{\fmtname \space<\fmtversion>}}
    \immediate
    \write16{\fmtname \space<\fmtversion>}
  \else\ifnum\patch@level>0
    \everyjob\expandafter{\the\everyjob
      \typeout{\fmtname \space<\fmtversion> patch level \patch@level}}
    \immediate
    \write16{\fmtname \space<\fmtversion> patch level \patch@level}
  \else
    \everyjob\expandafter{\the\everyjob
      \typeout{\fmtname \space<\fmtversion> pre-release\patch@level}}
    \immediate
    \write16{\fmtname \space<\fmtversion> pre-release\patch@level}
    \fi
  \fi
%</2ekernel>
%    \end{macrocode}
%
% \begin{macro}{\IncludeInRelease}
% \changes{v1.0n}{2015/01/07}{macro added}
% \changes{v1.0m}{2015/01/17}{modified with \cs{@currname}}
% \changes{v1.0o}{2015/01/19}{Optional argument}
% \changes{v1.0q}{2015/02/19}{Swap argument order}
%    \begin{macrocode}
%<*2ekernel|latexrelease>
\def\IncludeInRelease#1{\kernel@ifnextchar[%
  {\@IncludeInRelease{#1}}
  {\@IncludeInRelease{#1}[#1]}}
%    \end{macrocode}
%
% If a specific date has not been specified in |latexrelease|
% use `|#1|`. 
%    \begin{macrocode}
\def\@IncludeInRelease#1[#2]{\@IncludeInRele@se{#2}}
%    \end{macrocode}
%
%    \begin{macrocode}
\def\@IncludeInRele@se#1#2#3{%
  \toks@{[#1] #3}%
  \expandafter\ifx\csname\string#2+\@currname+IIR\endcsname\relax
    \ifnum\expandafter\@parse@version#1//00\@nil
          >\expandafter\@parse@version\fmtversion//00\@nil
      \GenericInfo{}{Skipping: \the\toks@}%
     \expandafter\expandafter\expandafter\@gobble@IncludeInRelease
    \else
      \GenericInfo{}{Applying: \the\toks@}%
      \expandafter\let\csname\string#2+\@currname+IIR\endcsname\@empty
    \fi
  \else
    \GenericInfo{}{Already applied: \the\toks@}%
    \expandafter\@gobble@IncludeInRelease
  \fi
}
%    \end{macrocode}
%
%    \begin{macrocode}
\long\def\@gobble@IncludeInRelease#1\EndIncludeInRelease{}
\let\EndIncludeInRelease\relax
%    \end{macrocode}
%
%    \begin{macrocode}
%</2ekernel|latexrelease>
%    \end{macrocode}
% \end{macro}
% \Finale
%
\endinput

\global\let\X@date=\@date

% Add the patch version if available.
   \long\def\Xdef#1#2#3\def#4#5{%
    \xdef\X@date{#2}%
    \xdef\patchdate{#5}%
    \endinput}%
   \InputIfFileExists{ltpatch.ltx}
    {\let\def\Xdef}{\global\let\includeltpatch\relax}
\endgroup

\ifx\@date\X@date
   \def\Xpatch{0}
   \ifx\patchdate\Xpatch\else
     \edef\@date{\@date\space Patch level \patchdate}
   \fi
\else
   \@warning{ltpatch.ltx does not match ltvers.dtx!}
   \let\includeltpatch\relax
\fi
\makeatother

 \pagenumbering{roman}
 \MaintainedByLaTeXTeam{latex}
 \maketitle
 \renewcommand\maketitle{}

 \tableofcontents

 \clearpage

 \pagenumbering{arabic}

%%%%%%%%%%%%%%%%%%%%%%%%%%%%%%%%%%%%%%%%%%%%%%%%%%%%%%%%%%%%

% Each of the following \DocInclude lines includes a file with extension
% .dtx. Each of these files may be typeset separately. For instance
% latex ltboxes.dtx
% will typeset the source of the LaTeX box commands.
%
% If this file is processed, each of these separate dtx files will be
% contained as a part of a single document. Using ltxdoc.cfg you can
% then optionally produce a combined index and/or change history for
% the entire source of the format file. Note that such a document will
% be quite large (about 555 pages).
%

 \DocInclude{ltdirchk} % System dependent initialisation

 \DocInclude{ltplain}  % LaTeX version of Knuth's plain.tex

 \DocInclude{ltvers}   % Current version date

 \DocInclude{ltdefns}  % Initial definitions.

 \DocInclude{ltalloc}  % Allocation of counters and others.

 \DocInclude{ltcntrl}  % Program control macros.

 \DocInclude{lterror}  % Error handling.

 \DocInclude{ltpar}    % Paragraphs.

 \DocInclude{ltspace}  % Spacing, line and page breaking.

 \DocInclude{ltlogos}  % Logos.

 \DocInclude{ltfiles}  % \input files and related commands

 \DocInclude{ltoutenc} % Output encoding interface

 \DocInclude{ltcounts} % Counters

 \DocInclude{ltlength} % Lengths

 \DocInclude{ltfssbas} % NFSS Base macros

 \DocInclude{ltfsstrc} % NFSS Tracing (and tracefnt.sty)

 \DocInclude{ltfsscmp} % NFSS1 Compatibility

 \DocInclude{ltfssdcl} % NFSS Declarative interface

 \DocInclude{ltfssini} % NFSS Initialisation

 \DocInclude{fontdef}  % fonttext.ltx/fontmath.ltx

 \DocInclude{preload}  % preload.ltx

 \DocInclude{ltfntcmd} % \textrm etc
 
 \DocInclude{ltpageno} % Page numbering

 \DocInclude{ltxref}   % Cross referencing

 \DocInclude{ltmiscen} % Miscellaneous environment definitions.

 \DocInclude{ltmath}   % Mathematics set up.

 \DocInclude{ltlists}  % List and related environments

 \DocInclude{ltboxes}  % Parbox and friends

 \DocInclude{lttab}    % Tabbing tabular and array

 \DocInclude{ltpictur} % Picture mode

 \DocInclude{ltthm}    % Theorem environments

 \DocInclude{ltsect}   % Sectioning

 \DocInclude{ltfloat}  % Floats

 \DocInclude{ltidxglo} % Index and Glossary

 \DocInclude{ltbibl}   % Bibliography

 \DocInclude{ltpage}   % \pagestyle \raggedbottom \sloppy

 \DocInclude{ltoutput} % Output routine

 \DocInclude{ltclass}  % Package & Class interface

 \DocInclude{lthyphen} % Hyphenation (hyphen.ltx).

 \DocInclude{ltluatex} % Luatex support

 \DocInclude{ltfinal}  % Last minute initialisations and dump

 \includeltpatch       % Corrections distributed after the full release

% Stop here if ltxdoc.cfg says \AtEndOfClass{\OnlyDescription}
\StopEventually{\end{document}}

\clearpage
\pagestyle{headings}

% Make TeX shut up.
\hbadness=10000
\newcount\hbadness
\hfuzz=\maxdimen

\typeout{%
  \string # Produce change log with^^J%
  makeindex -s gglo.ist -o source2e.gls source2e.glo}


\PrintChanges

\clearpage

% makeindex needs a symbol between the parts of composite page numbers
% but we dont want one, so:
\typeout{%
  \string # Produce index with^^J%
  makeindex -s source2e.ist source2e.idx}

\begingroup
\def\endash{--}
\catcode`\-\active
\def-{\futurelet\temp\indexdash}
\def\indexdash{\ifx\temp-\endash\fi}

\PrintIndex
\endgroup

% Make sure that the index is not printed twice
% (ltxdoc.cfg might have a second \PrintIndex command)
\let\PrintChanges\relax
\let\PrintIndex\relax

\end{document}


%%%%%%%%%%%%%%%%%%%%%%%%%%%%%%%%%%%%%%%%%%%%%%%%%%%%%%%%%%%%%%

To use this file to produce a fully indexed source code
you need to execute the following (or equivalent) commands:

   latex source2e.tex

   makeindex -s source2e.ist source2e.idx
   makeindex -s gglo.ist -o source2e.gls source2e.glo

   latex source2e.tex
   latex source2e.tex


The makeindex style source2e.ist is used in place of the usual
doc gind.ist to ensure that I is used in the sequence I J K
not I II II, which would be the default makeindex behaviour.

The third run with latex is only required to get the table of
contents entries for the change log and index. You may speed things up
by using the \includeonly mechanism so as not to typeset the source
files on the second run. This involves changing the file
ltxdoc.cfg
between the latex runs.

The following unix script automates this.
  (It could easily be ported to scripts for DOS or VMS,
   rm is ReMove a file, and echo "..." > file writes ... to "file".)


After this script (after the second ==============) is a similar script
that will produce the documentation for all the files in the base
distribution that are *not* included in source2e.dvi. This second script
was requested, but before using it, beware it will take a long time!
It may however be modified as required, eg to not typeset the fdd files
or whatever...

==============
#!/bin/sh

rm  -f source2e.gls source2e.ind source2e.toc

# First run: 
# Create new standard ltxdoc.cfg file
# Pass the (possibly empty) list of arguments supplied on the
# command line to article class.
#
# If you use A4 paper, running this script with argument
#    a4paper
# may save about 30 pages.
#
echo "\PassOptionsToClass{$*}{article}" > ltxdoc.cfg


# Now LaTeX the file with this cfg file.
#
latex source2e.tex


# Make the Change log and Glossary.
#
makeindex -s source2e.ist source2e.idx
makeindex -s gglo.ist -o source2e.gls source2e.glo


# Second run: append \includeonly{} to ltxdoc.cfg to speed up things
# (this run needed only to get changes and index listed in .toc file)
#
# Note that the index will not be made incorrect by the insertion
# of the table of contents as the front matter uses a different page
# numbering scheme.
#
echo "\includeonly{}" >> ltxdoc.cfg

latex source2e.tex


# Third and final run, to put everything together.
# First restore the cfg file:
#
echo "\PassOptionsToClass{$*}{article}" > ltxdoc.cfg
latex source2e.tex


==============
#!/bin/sh

# Running this script will process all the dtx fdd and *guide.tex
# and ltnews*.tex files in the LaTeX distribution, except the dtx
# files included in source2e.tex. 
# (The shell first script in the comments of source2e.tex will
#  process those.)

# Any command line arguments (eg a4paper) are taken as options to the
# article class.

# This script is likely to take ages!

echo "\PassOptionsToClass{$*}{article}"                  > ltxdoc.cfg
echo "\batchmode"                                       >> ltxdoc.cfg

# The next four lines produce full indexes and change logs
# you may not want those.
echo "\AtBeginDocument{\RecordChanges}"                 >> ltxdoc.cfg
echo "\AtEndDocument{\PrintChanges}"                    >> ltxdoc.cfg
echo "\AtBeginDocument{\CodelineIndex\EnableCrossrefs}" >> ltxdoc.cfg
echo "\AtEndDocument{\PrintIndex}"                      >> ltxdoc.cfg

# If you do not want any code listings, just documentation, then instead
# of the above four lines, uncomment the following:
# echo "\AtBeginDocument{\OnlyDescription}"                >> ltxdoc.cfg

echo "\PassOptionsToClass{$*}{article}"                  > ltxguide.cfg
echo "\batchmode"                                       >> ltxguide.cfg

cp ltxguide.cfg ltnews.cfg


for i in *dtx *fdd *guide.tex ltnews*.tex
do
B=`basename $i .dtx`

if (grep "Include{$B}" source2e.tex >/dev/null ; )
then
echo In source2e: $i
else
echo latex $i
  if (latex $i > /dev/null) 
  then
    echo latex $i
    latex $i > /dev/null
    echo makeindex -s gind.ist $B.idx
    makeindex -s gind.ist $B.idx > /dev/null 2> /dev/null
    echo makeindex -s gglo.ist -o $B.gls $B.glo
    makeindex -s gglo.ist -o $B.gls $B.glo > /dev/null 2> /dev/null
    echo latex $i
    latex $i > /dev/null
  else
    echo "!!! LaTeX ERROR: $i. (See $B.log.)"
  fi
fi

done
