% \iffalse meta-comment
%
% Copyright 1993-2015
% The LaTeX3 Project and any individual authors listed elsewhere
% in this file.
%
% This file is part of the LaTeX base system.
% -------------------------------------------
%
% It may be distributed and/or modified under the
% conditions of the LaTeX Project Public License, either version 1.3c
% of this license or (at your option) any later version.
% The latest version of this license is in
%    http://www.latex-project.org/lppl.txt
% and version 1.3c or later is part of all distributions of LaTeX
% version 2005/12/01 or later.
%
% This file has the LPPL maintenance status "maintained".
%
% The list of all files belonging to the LaTeX base distribution is
% given in the file `manifest.txt'. See also `legal.txt' for additional
% information.
%
% The list of derived (unpacked) files belonging to the distribution
% and covered by LPPL is defined by the unpacking scripts (with
% extension .ins) which are part of the distribution.
%
% \fi
%
% \iffalse
%%% From File: ltcntrl.dtx
%<*driver>
% \fi
\ProvidesFile{ltcntrl.dtx}
             [2014/04/21 v1.0h LaTeX Kernel (program control)]
% \iffalse
\documentclass{ltxdoc}
\GetFileInfo{ltcntrl.dtx}
\title{\filename}
\date{\filedate}
 \author{%
  Johannes Braams\and
  David Carlisle\and
  Alan Jeffrey\and
  Leslie Lamport\and
  Frank Mittelbach\and
  Chris Rowley\and
  Rainer Sch\"opf}
\begin{document}
 \MaintainedByLaTeXTeam{latex}
 \maketitle
 \DocInput{\filename}
\end{document}
%</driver>
% \fi
%
% \CheckSum{176}
%
% \changes{v1.0a}{1994/05/16}{(ASAJ) Split from ltinit.dtx.}
% \changes{v1.0b}{1994/11/17}
%         {\cs{@tempa} to \cs{reserved@a}}
% \changes{v1.0c}{1994/11/28}
%         {Documentation improvements}
%
% \section{Program control structure}
%
% This section defines a number of control structure macros, such as
% while-loops and for-loops.
%
% \StopEventually{}
%
% \begin{oldcomments}
%    \begin{macrocode}
%<*2ekernel>
\message{control,}
%    \end{macrocode}
%
% \@whilenum TEST \do {BODY}
% \@whiledim TEST \do {BODY}  : These implement the loop
%           while  TEST  do  BODY  od
%     where  TEST  is a TeX \ifnum or \ifdim test, respectively.
%     They are optimized for the normal case of TEST initially false.
%
% \@whilesw SWITCH \fi {BODY} : Implements the loop
%               while SWITCH do BODY od
%     Optimized for normal case of SWITCH initially false.
%
% \@for NAME := LIST \do {BODY} : Assumes that LIST expands to A1,A2,
%      ... ,An .
%      Executes  BODY  n  times, with  NAME = Ai  on the i-th iteration.
%      Optimized for the normal case of n = 1.  Works for n=0.
%
% \@tfor NAME := LIST \do {BODY}
%      if, before expansion, LIST = T1 ... Tn  where each Ti is a
%      token or {...}, then executes  BODY  n  times, with  NAME = Ti
%      on the i-th iteration.  Works for n=0.
%
%  NOTES: 1. These macros use no \@temp sequences.
%         2. These macros do not work if the body contains anything that
%         looks syntactically to TeX like an improperly balanced \if
%         \else \fi.
%
% \@whilenum TEST \do {BODY} ==
%  BEGIN
%    if  TEST
%      then  BODY
%            \@iwhilenum{TEST \relax BODY}
%  END
%
% \@iwhilenum {TEST BODY} ==
%  BEGIN
%    if  TEST
%      then  BODY
%            \@nextwhile = def(\@iwhilenum)
%      else  \@nextwhile = def(\@whilenoop)
%    fi
%    \@nextwhile {TEST BODY}
%  END
%
% \@whilesw SWITCH \fi {BODY} ==
%  BEGIN
%    if SWITCH
%      then BODY
%           \@iwhilesw {SWITCH BODY}\fi
%    fi
%  END
%
% \@iwhilesw {SWITCH BODY} \fi ==
%  BEGIN
%    if SWITCH
%      then BODY
%           \@nextwhile = def(\@iwhilesw)
%      else \@nextwhile = def(\@whileswnoop)
%    fi
%    \@nextwhile {SWITCH BODY} \fi
%  END
%
% \end{oldcomments}
%
% \begin{macro}{\@whilenoop}
% \begin{macro}{\@whilenum}
% \begin{macro}{\@iwhilenum}
% \changes{v1.0f}{1995/07/09}{Reimplemented using Kabelschacht method}
% \changes{v1.0g}{1995/08/16}{Removed \cs{@whilenoop}}
% \changes{v1.0g}{1995/08/16}{Made defs long}
%    \begin{macrocode}
\long\def\@whilenum#1\do #2{\ifnum #1\relax #2\relax\@iwhilenum{#1\relax
     #2\relax}\fi}
\long\def\@iwhilenum#1{\ifnum #1\expandafter\@iwhilenum
         \else\expandafter\@gobble\fi{#1}}
%    \end{macrocode}
% \end{macro}
% \end{macro}
% \end{macro}
%
% \begin{macro}{\@whiledim}
% \begin{macro}{\@iwhiledim}
% \changes{v1.0f}{1995/07/09}{Reimplemented using Kabelschacht method}
% \changes{v1.0g}{1995/08/16}{Removed \cs{@whilenoop}}
% \changes{v1.0g}{1995/08/16}{Made defs long}
%    \begin{macrocode}
\long\def\@whiledim#1\do #2{\ifdim #1\relax#2\@iwhiledim{#1\relax#2}\fi}
\long\def\@iwhiledim#1{\ifdim #1\expandafter\@iwhiledim
        \else\expandafter\@gobble\fi{#1}}
%    \end{macrocode}
% \end{macro}
% \end{macro}
%
% \begin{macro}{\@whileswnoop}
% \begin{macro}{\@whilesw}
% \begin{macro}{\@iwhilesw}
% \changes{v1.0f}{1995/07/09}{Reimplemented using Kabelschacht method}
% \changes{v1.0g}{1995/08/16}{Removed \cs{@whileswnoop}}
%    \begin{macrocode}
\long\def\@whilesw#1\fi#2{#1#2\@iwhilesw{#1#2}\fi\fi}
\long\def\@iwhilesw#1\fi{#1\expandafter\@iwhilesw
         \else\@gobbletwo\fi{#1}\fi}
%    \end{macrocode}
% \end{macro}
% \end{macro}
% \end{macro}
%
% \begin{oldcomments}
%
% \@for NAME := LIST \do {BODY} ==
%    BEGIN \@forloop expand(LIST),\@nil,\@nil \@@ NAME {BODY} END
%
% \@forloop CAR, CARCDR, CDRCDR \@@ NAME {BODY} ==
%   BEGIN
%     NAME = CAR
%     if def(NAME) = def(\@nnil)
%       else BODY;
%            NAME = CARCDR
%            if def(NAME) = def(\@nnil)
%              else BODY
%                   \@iforloop CDRCDR \@@ NAME \do {BODY}
%            fi
%     fi
%   END
%
% \@iforloop CAR, CDR \@@ NAME {BODY} =
%     NAME = CAR
%     if def(NAME) = def(\@nnil)
%        then  \@nextwhile = def(\@fornoop)
%        else  BODY ;
%              \@nextwhile = def(\@iforloop)
%     fi
%     \@nextwhile name cdr {body}
%
% \@tfor NAME := LIST \do {BODY}
%    =  \@tforloop LIST \@nil \@@ NAME {BODY}
%
% \@tforloop car cdr \@@ name {body} =
%     name = car
%     if def(name) = def(\@nnil)
%        then  \@nextwhile == \@fornoop
%        else  body ;
%              \@nextwhile == \@forloop
%     fi
%     \@nextwhile name cdr {body}
% \end{oldcomments}
%
% \begin{macro}{\@nnil}
%    \begin{macrocode}
\def\@nnil{\@nil}
%    \end{macrocode}
% \end{macro}
%
% \begin{macro}{\@empty}
%    \begin{macrocode}
\def\@empty{}
%    \end{macrocode}
% \end{macro}
%
% \begin{macro}{\@fornoop}
% \changes{v1.0g}{1995/08/16}{Made defs long}
% \changes{v1.0h}{2007/08/06}{Really make defs long}
%    \begin{macrocode}
\long\def\@fornoop#1\@@#2#3{}
%    \end{macrocode}
% \end{macro}
%
% \begin{macro}{\@for}
% \changes{v1.0d}{1995/04/24}
%      {Dont expand second argument with \cs{edef}: /1317 (DPC)}
%    \begin{macrocode}
\long\def\@for#1:=#2\do#3{%
  \expandafter\def\expandafter\@fortmp\expandafter{#2}%
  \ifx\@fortmp\@empty \else
    \expandafter\@forloop#2,\@nil,\@nil\@@#1{#3}\fi}
%    \end{macrocode}
% \end{macro}
%
% \begin{macro}{\@forloop}
% \changes{v1.0g}{1995/08/16}{Made defs long}
%    \begin{macrocode}
\long\def\@forloop#1,#2,#3\@@#4#5{\def#4{#1}\ifx #4\@nnil \else
       #5\def#4{#2}\ifx #4\@nnil \else#5\@iforloop #3\@@#4{#5}\fi\fi}
%    \end{macrocode}
% \end{macro}
%
% \begin{macro}{\@iforloop}
% \changes{v1.0f}{1995/07/09}{Reimplemented using Kabelschacht method}
% \changes{v1.0g}{1995/08/16}{Made defs long}
%    \begin{macrocode}
\long\def\@iforloop#1,#2\@@#3#4{\def#3{#1}\ifx #3\@nnil
       \expandafter\@fornoop \else
      #4\relax\expandafter\@iforloop\fi#2\@@#3{#4}}
%    \end{macrocode}
% \end{macro}
%
% \begin{macro}{\@tfor}
% \changes{LaTeX209}{1991/10/17}
%         {(Rms) \cs{xdef} replaced by \cs{def}
%          (See FMi's array.doc)}
% \changes{v1.0c}{1994/03/13}
%         {(DPC) Add \cs{@tf@r} so a single group is
%           correctly treated.}
% \changes{v1.0f}{1995/07/09}{Reimplemented using Kabelschacht method}
% \changes{v1.0g}{1995/08/16}{Made defs long}
%    \begin{macrocode}
\def\@tfor#1:={\@tf@r#1 }
\long\def\@tf@r#1#2\do#3{\def\@fortmp{#2}\ifx\@fortmp\space\else
    \@tforloop#2\@nil\@nil\@@#1{#3}\fi}
\long\def\@tforloop#1#2\@@#3#4{\def#3{#1}\ifx #3\@nnil
       \expandafter\@fornoop \else
      #4\relax\expandafter\@tforloop\fi#2\@@#3{#4}}
%    \end{macrocode}
% \end{macro}
%
% \begin{macro}{\@break@tfor}
% Break out of a |\@tfor| loop. This should be called \emph{inside}
% the scope of an |\if|. See |\@iffileonpath| for an example.
% \changes{v1.0l}{1994/05/02}{Macro added (from ltfiles.dtx)}
% \changes{v1.0g}{1995/08/16}{Made long}
%    \begin{macrocode}
\long\def\@break@tfor#1\@@#2#3{\fi\fi}
%    \end{macrocode}
% \end{macro}
%
% \begin{macro}{\@removeelement}
%    Removes an element from a comma-separated list and puts it into
%    a control sequence, called as
%    |\@removeelement{|\meta{element}|}{|\meta{list}|}{|\meta{cs}|}|.
%    Due to the implementation method the \meta{element} is not allowed
%    to contain braces.
%    \begin{macrocode}
\def\@removeelement#1#2#3{%
  \def\reserved@a##1,#1,##2\reserved@a{##1,##2\reserved@b}%
  \def\reserved@b##1,\reserved@b##2\reserved@b{%
    \ifx,##1\@empty\else##1\fi}%
  \edef#3{%
    \expandafter\reserved@b\reserved@a,#2,\reserved@b,#1,\reserved@a}}
%    \end{macrocode}
% \end{macro}
%
%
% \changes{v1.0e}{1995/04/29}{Removed unused defs for
%              \cs{@setprotect} and \cs{@resetprotect}}
% \changes{v1.0e}{1995/04/29}{Moved init of \cs{protect}
%              to ltdefns.dtx}
%    \begin{macrocode}
%</2ekernel>
%    \end{macrocode}
%
%\Finale
\endinput
